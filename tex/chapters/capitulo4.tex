\chapter[TRANSFORMACIONES LINEALES]{TRANSFORMACIONES \\ LINEALES}
%\startcontents
\printchaptertableofcontents

Las transformaciones lineales son un pilar esencial en el vasto terreno de las matemáticas, particularmente en el ámbito de álgebra lineal. Para comprender la magnitud de su influencia, es imperativo sumergirse en la esencia misma de estas transformaciones y explorar sus aplicaciones en diversos campos.

Una transformación lineal es, en esencia, una función matemática que preserva la estructura algebraica de los vectores. Al aplicar una transformación lineal a la combinación lineal de dos vectores, los resultados son análogos a la combinación lineal de las transformaciones individuales de cada vector. Este comportamiento de preservación de la linealidad es una característica clave que distingue a las transformaciones lineales y las eleva a una posición central en la teoría algebraica.

%En el fondo de estas transformaciones reside la preservación de las operaciones fundamentales: la suma y la multiplicación por escalares. Cuando aplicamos una transformación lineal a la suma de dos vectores, el resultado es la suma de las transformaciones lineales de cada vector individual. Similarmente, la multiplicación de un vector por un escalar antes de la transformación lineal es equivalente a multiplicar la transformación lineal por ese mismo escalar. Estas propiedades fundamentales son cruciales para entender la coherencia y la estructura que las transformaciones lineales aportan al álgebra lineal.

La aplicabilidad de las transformaciones lineales se extiende a diversos campos de estudio. En el ámbito de la resolución de sistemas de ecuaciones lineales, las transformaciones lineales ofrecen herramientas poderosas para entender y abordar problemas complejos. Además, en el análisis de estructuras algebraicas, como espacios vectoriales y grupos, las transformaciones lineales juegan un papel vital al proporcionar una lente única para examinar las propiedades intrínsecas de estos objetos matemáticos.

Al profundizar en las propiedades y aplicaciones de las transformaciones lineales, se revela un panorama rico y complejo. Desde la diagonalización de matrices hasta la representación canónica de formas cuadráticas, las transformaciones lineales ofrecen un marco conceptual robusto para abordar una variedad de problemas matemáticos.

Las transformaciones lineales no solo son conceptos abstractos en el ámbito de álgebra lineal, sino que constituyen la esencia misma de la coherencia algebraica. Su estudio no solo enriquece nuestra comprensión de las estructuras matemáticas, sino que también desbloquea un conjunto diverso de herramientas analíticas con aplicaciones prácticas significativas.

\section{Definición y ejemplos}

\begin{definition}\label{def:operatorlineal}
    Sean $V$ y $W$ dos espacios vectoriales sobre $K = \RR$. Se dice que una función
    \begin{align*}
        T: V & \longrightarrow W \\
        \mathbb{v} & \longmapsto T\mathbb{v} = \mathbb{w}
    \end{align*}
    es una transformación lineal de $V$ en $W$ si cumple que para cada $\mathbb{u}$, $\mathbb{v} \in V$ y $\alpha \in K$\infoBulle{A una transformación lineal también se le puede llamar operador lineal.}
    \begin{enumerate}[label=\roman*)]
        \item $T(\mathbb{u} + \mathbb{v}) = T\mathbb{u} + T\mathbb{v}$.
        \item $T(\alpha \mathbb{u}) = \alpha T\mathbb{u}$.
    \end{enumerate}
\end{definition}

\begin{notation}
    Se escriben indistintamente $T\mathbb{v}$ y $T(\mathbb{v})$. Denotan lo mismo; las dos se leen “$T$ de $\mathbb{v}$”. Esto es análogo a la notación funcional $f(x)$, que se lee “$f$ de $x$”.
\end{notation}

\begin{observation}
    La notación $T: V \longrightarrow W$ indica que $T$ toma el espacio vectorial real $V$ y lo lleva al espacio vectorial real $W$; esto es, $T$ es una función con $V$ como su dominio y un subconjunto de $W$ como su imagen.
\end{observation}

\begin{example}
    Veamos si
    \begin{align*}
        T: \RR[2] & \longrightarrow \RR[2] \\
        \begin{pmatrix}
            x \\
            y
        \end{pmatrix} & \longmapsto T \begin{pmatrix}
            x \\
            y
        \end{pmatrix} = \begin{pmatrix*}[r]
            x \\
            -y
        \end{pmatrix*}
    \end{align*}
    es una transformación lineal. En este caso, $T: V \longrightarrow W$ es $T: \RR[2] \longrightarrow \RR[2]$, además $\RR[2]$ es un espacio vectorial sobre $\RR$. Notemos que $T$ admite una interpretación geométrica sencilla como se muestra en la figura \ref{HDFDDFHUYFGHFUYGFFGUY}, así se comprueba que $T$ es función. Ahora se tiene\sideFigure[\label{HDFDDFHUYFGHFUYGFFGUY}Transformación de reflexión respecto al eje $x$]{
    \begin{tikzpicture}[scale=0.83]
        \draw[thick,-Stealth] (-1,0) -- (5,0);
        \draw[thick,Stealth-Stealth] (0,-5.5) -- (0,5.5);
        \draw[dash pattern=on 3pt off 3pt] (0,4) node[left] {$y$} -- (4,4) -- (4,-4) -- (0,-4) node[left] {$-y$};
        \draw[thick,-latex] (0,0) -- (4,4) node[above] {$\begin{pmatrix}
            x \\
            y
        \end{pmatrix}$};
        \draw[thick,-latex] (0,0) -- (4,-4) node[below] {$\begin{pmatrix*}[r]
            x \\
            -y
        \end{pmatrix*}$};
        \node at (0,0) [below left] {$\mathbb{0}$};
        \node at (4,0) [above right] {$x$};
        \node at (2,2) [above,rotate=45] {$\mathbb{u}$};
        \node at (2,-2) [below,rotate=-45] {$\mathbb{w} = T\mathbb{u}$};
    \end{tikzpicture}
    }
    \begin{enumerate}[label=\roman*)]
        \item Sea $\mathbb{u}$, $\mathbb{v} \in \RR[2]$ con $\mathbb{u} = \begin{pmatrix}
            x_1 \\
            y_1
        \end{pmatrix}$, $\mathbb{v} = \begin{pmatrix}
            x_2 \\
            y_2
        \end{pmatrix}$. Entonces
        \begin{align*}
            T(\mathbb{u} + \mathbb{v}) & = T\left( \begin{pmatrix}
                x_1 \\
                y_1
            \end{pmatrix} + \begin{pmatrix}
                x_2 \\
                y_2
            \end{pmatrix} \right) \\
            & = T \begin{pmatrix}
                x_1 + x_2 \\
                y_1 + y_2
            \end{pmatrix} \\
            & = \begin{pmatrix}
                x_1 + x_2 \\
                -(y_1 + y_2)
            \end{pmatrix} \\
            & = \begin{pmatrix}
                x_1 + x_2 \\
                -y_1 + (-y_2)
            \end{pmatrix} \\
            & = \begin{pmatrix*}[r]
                x_1 \\
                -y_1
            \end{pmatrix*} + \begin{pmatrix*}[r]
                x_2 \\
                -y_2
            \end{pmatrix*} \\
            & = T\mathbb{u} + T\mathbb{v}
        \end{align*}
        Por tanto $T(\mathbb{u} + \mathbb{v}) = T\mathbb{u} + T\mathbb{v}$.\newpage
        \item Sea $\mathbb{u} \in \RR[2]$ con $\mathbb{u} = \begin{pmatrix}
            x_1 \\
            y_1
        \end{pmatrix}$ y $\alpha \in \RR$. Entonces
        \begin{align*}
            T(\alpha \mathbb{u}) & = T \left( \alpha \begin{pmatrix}
                x_1 \\
                y_1
            \end{pmatrix} \right) \\
            & = T \begin{pmatrix}
                \alpha x_1 \\
                \alpha y_1
            \end{pmatrix} \\
            & = \begin{pmatrix*}[r]
                \alpha x_1 \\
                - \alpha y_1
            \end{pmatrix*} \\
            & = \alpha \begin{pmatrix*}[r]
                x_1 \\
                -y_1
            \end{pmatrix*} \\
            & = \alpha T \mathbb{u}
        \end{align*}
        Por tanto, $T(\alpha \mathbb{u}) = \alpha T\mathbb{u}$.
    \end{enumerate}
    Por tanto, $T$ es una transformación lineal de $\RR[2]$ a $\RR[2]$.
\end{example}

\begin{example}\label{tl:rotacion}
    Supongamos que el vector $\mathbb{v}$ se rota un ángulo $\theta$ (medido en grados o radianes) en sentido contrario al de las manecillas del reloj. Llamemos a este nuevo vector rotado $\mathbb{v}'$. Entonces como se ve en la figura \ref{fig:rotado1}, si $r$ denota la longitud de $\mathbb{v}$, entonces\sideFigure[\label{fig:rotado1}Transformación de rotación]{
    \begin{center}
        \begin{tikzpicture}[scale=0.76]
            \coordinate (A) at (0,0);
            \draw[-Stealth,thick] (-1,0) -- (5.5,0) node[right] {$x$} coordinate (B);
            \draw[-Stealth,thick] (0,-1) -- (0,5.5) node[above] {$y$};
            \draw[dash pattern=on 3pt off 3pt,thin] (3,0) -- (3,4) -- (0,4);
            \draw[dash pattern=on 3pt off 3pt,thin] (4.2,0) -- (4.2,1.6) -- (0,1.6);
            \draw[-latex,thick] (0,0) -- (3,4) coordinate(D);
            \draw[-latex,thick] (0,0) -- (4.2,1.6) coordinate(C);
            \node at (3,4) [right] {$\mathbb{v}' = \begin{pmatrix}
                x' \\
                y'
            \end{pmatrix}$};
            \node at (4.2,1.6) [right] {$\mathbb{v} = \begin{pmatrix}
                x \\
                y
            \end{pmatrix}$};
            \pic[draw, -latex, "$\alpha$", angle eccentricity=1.3,angle radius=1cm] {angle = B--A--C};
            \pic[draw, -latex, "$\theta$", angle eccentricity=1.3,angle radius=0.8cm] {angle = C--A--D};
            \pic[draw, -latex, "$\theta + \alpha$", angle eccentricity=1.3,angle radius=1.55cm] {angle = B--A--D};
            \node at (0,0) [below left] {$\mathbb{0}$};
        \end{tikzpicture}
    \end{center}
    }
    \marginElement{\justify
    \begin{center}
        \colorbox{gray!20}{\parbox[c]{\dimexpr\linewidth-3pt-2\fboxsep-2\fboxrule}{
            La transformación lineal de rotación es un proceso matemático que gira un objeto alrededor de un punto fijo. En un espacio bidimensional, la rotación se puede describir mediante matrices de rotación como
            $$\begin{pmatrix}
                x' \\
                y'
            \end{pmatrix} = \begin{bmatrix*}[r]
                \cos \theta & - \sen \theta \\
                \sen \theta & \cos \theta
            \end{bmatrix*} \begin{bmatrix}
                x \\
                y
            \end{bmatrix}$$
            Estas matrices aplican operaciones lineales para cambiar las coordenadas del objeto y lograr la rotación deseada. La trigonometría es fundamental en la formulación de estas matrices, y la aplicación repetida de rotaciones puede dar lugar a transformaciones más complejas.
        }}
    \end{center}
    }
    \begin{equation}
        x = r \cos(\alpha), \quad y = r \sen(\alpha)
    \end{equation}
    y
    \begin{equation}
        x' = r \cos(\theta + \alpha), \quad y' = r \sen(\theta + \alpha)
    \end{equation}
    Entonces
    \begin{align*}
        x' & = r \big( \cos(\theta) \cos(\alpha) - \sen(\theta) \sen(\alpha) \big) \\
        & = r \cos(\theta) \cos(\alpha) - r \sen(\theta) \sen(\alpha) \\
        & = x \cos(\theta) - y \sen(\theta)
    \end{align*}
    De manera análoga,
    \begin{align*}
        y' & = \big( \cos(\theta) \sen(\alpha) + \cos(\alpha) \sen(\theta) \big) \\
        & = r \cos(\theta) \sen(\alpha) + r \cos(\alpha) \sen(\theta) \\
        & = y \cos(\theta) + x \sen(\theta)
    \end{align*}
    Así pues, sea
    \begin{align*}
        T: \RR[2] & \longrightarrow \RR[2] \\
        \begin{pmatrix}
            x \\
            y
        \end{pmatrix} & \longmapsto T \begin{pmatrix}
            x \\
            y
        \end{pmatrix} = \begin{pmatrix}
            x \cos \theta - y \sen \theta \\
            x \sen \theta + y \cos \theta
        \end{pmatrix}
    \end{align*}
    con $0 \leq \theta < 2 \pi$ y $r > 0$ donde $r = \sqrt{x^2+y^2}$. Probemos que $T$ es transformación lineal:
    \begin{enumerate}[label=\roman*)]
        \item Sea $\mathbb{u}$, $\mathbb{v} \in \RR[2]$ con $\mathbb{u} = \begin{pmatrix}
            x \\
            y
        \end{pmatrix}$, $\mathbb{v} = \begin{pmatrix}
            x' \\
            y'
        \end{pmatrix}$. Entonces
        \begin{align*}
            T(\mathbb{u} + \mathbb{v}) & = T \begin{pmatrix}
                x + x' \\
                y + y'
            \end{pmatrix} \\
            & = \begin{pmatrix}
                (x + x') \cos \theta - (y + y') \sen \theta \\
                (x + x') \sen \theta + (y + y') \cos \theta
            \end{pmatrix} \\
            & = \begin{pmatrix}
                x \cos \theta + x' \cos \theta - y \sen \theta - y' \sen \theta \\
                x \sen \theta + x' \sen \theta + y \cos \theta + y' \cos \theta
            \end{pmatrix} \\
            & = \begin{pmatrix}
                x \cos \theta - y \sen \theta \\
                x \sen \theta + y \cos \theta
            \end{pmatrix} + \begin{pmatrix}
                x' \cos \theta - y' \sen \theta \\
                x' \sen \theta - y' \cos \theta
            \end{pmatrix} \\
            & = T\mathbb{u} + T\mathbb{v}
        \end{align*}
        Por tanto $T(\mathbb{u} + \mathbb{v}) = T\mathbb{u} + T\mathbb{v}$.
        \item Se deja como ejercicio al lector.
    \end{enumerate}
    Por tanto, $T$ es una transformación lineal de $\RR[2]$ a $\RR[2]$.
\end{example}

\begin{example}
    Veamos si
    \begin{align*}
        T: \RR[2] & \longrightarrow \RR[3] \\
        \begin{pmatrix}
            x \\
            y
        \end{pmatrix} & \longmapsto T \begin{pmatrix}
            x \\
            y
        \end{pmatrix} = \begin{pmatrix}
            x + y \\
            x - y \\
            2y
        \end{pmatrix}
    \end{align*}
    es una transformación lineal.
    \begin{enumerate}[label=\roman*)]
        \item Sea $\mathbb{u}$, $\mathbb{v} \in \RR[2]$ con $\mathbb{u} = \begin{pmatrix}
            x_1 \\
            y_1
        \end{pmatrix}$, $\mathbb{v} =  \begin{pmatrix}
            x_2 \\
            y_2
        \end{pmatrix}$. Entonces
        \begin{align*}
            T(\mathbb{u} + \mathbb{v}) & = T \left( \begin{pmatrix}
                x_1 \\
                y_1
            \end{pmatrix} + \begin{pmatrix}
                x_2 \\
                y_2
            \end{pmatrix} \right) \\
            & = T \begin{pmatrix}
                x_1 + x_2 \\
                y_1 + y_2
            \end{pmatrix} \\
            & = \begin{pmatrix}
                x_1 + x_2 + y_1 + y_2 \\
                x_1 + x_2 - (y_1 + y_2) \\
                2(y_1 + y_2)
            \end{pmatrix} \\
            & = \begin{pmatrix}
                x_1 + y_1 + x_2 + y_2 \\
                x_1 - y_1 + x_2 - y_2 \\
                2y_1 + 2y_2
            \end{pmatrix} \\
            & = \begin{pmatrix}
                x_1 + y_1 \\
                x_1 - y_1 \\
                2y_1
            \end{pmatrix} + \begin{pmatrix}
                x_2 + y_2 \\
                x_2 - y_2 \\
                2y_2
            \end{pmatrix} \\
            & = T\mathbb{u} + T\mathbb{v}
        \end{align*}
        Por tanto $T(\mathbb{u} + \mathbb{v}) = T\mathbb{u} + T\mathbb{v}$.
        \item Sea $\mathbb{u} \in \RR[2]$ con $\mathbb{u} = \begin{pmatrix}
            x_1 \\
            y_1
        \end{pmatrix}$ y $\alpha \in \RR$. Entonces
        \begin{align*}
            T(\alpha \mathbb{u}) & = T \left( \alpha \begin{pmatrix}
                x_1 \\
                y_1
            \end{pmatrix} \right) \\
            & = T \begin{pmatrix}
                \alpha x_1 \\
                \alpha y_1
            \end{pmatrix} \\
            & = \begin{pmatrix}
                \alpha x_1 + \alpha y_1 \\
                \alpha x_1 - \alpha y_1 \\
                2 \alpha y_1
            \end{pmatrix} \\
            & = \alpha \begin{pmatrix}
                x_1 + y_1 \\
                x_1 - y_1 \\
                2 y_1
            \end{pmatrix} \\
            & = \alpha T \mathbb{u}
        \end{align*}
        Por tanto, $T(\alpha \mathbb{u}) = \alpha T\mathbb{u}$.
    \end{enumerate}
    Por tanto, $T$ es una transformación lineal de $\RR[2]$ a $\RR[3]$.
\end{example}

\begin{definition}
    A la transformación lineal dada por $\mathcal{O} \mathbb{u} = \mathbb{0}_{W}$, $\forall \mathbb{u} \in V$ siendo $\mathcal{O}:V \longrightarrow W$, se le llama transformación cero.
\end{definition}

\begin{definition}
    A la transformación lineal dada por $I \mathbb{u} = \mathbb{u}$, $\forall \mathbb{u} \in V$ siendo $I:V \longrightarrow V$, se le llama transformación identidad.
\end{definition}

\begin{example}
    Sea $T: \RR[n] \longrightarrow \RR[m]$ definida como
    $$T\mathbb{x} = A\mathbb{x}$$
    siendo $A \in \matrizmn$. Verifique que $T$ es una transformación lineal. \\
    \solucion Sea
    \begin{align*}
        T: \RR[n] & \longrightarrow \RR[m] \\
        \mathbb{x} & \longmapsto T\mathbb{x} = A\mathbb{x}
    \end{align*}
    Probemos que $T$ es una transformación lineal:
    \begin{enumerate}[label=\roman*)]
        \item Sea $\mathbb{x}$, $\mathbb{y} \in \RR[n]$. Entonces
        \begin{align*}
            T(\mathbb{x} + \mathbb{y}) & = A(\mathbb{x} + \mathbb{y}) \\
            & = A\mathbb{x} + \mathbb{y} \\
            & = T\mathbb{x} + T\mathbb{y}
        \end{align*}
        Por tanto $T(\mathbb{x} + \mathbb{y}) = T\mathbb{x} + T\mathbb{y}$.
        \item Sea $\mathbb{x} \in \RR[n]$ y $\alpha \in \RR$. Entonces
        \begin{align*}
            T(\alpha \mathbb{x}) & = A(\alpha \mathbb{x}) \\
            & = \alpha A \mathbb{x} \\
            & = \alpha T \mathbb{x}
        \end{align*}
        Por tanto, $T(\alpha \mathbb{x}) = \alpha T\mathbb{x}$.
    \end{enumerate}
    Por tanto, $T$ es una transformación lineal de $\RR[n]$ a $\RR[m]$.
\end{example}

\begin{example}
    Sea
    \begin{align*}
        T:C[0,  1] & \longrightarrow \RR \\
        f & \longmapsto T_{f} = \int_0^1 f(x) dx
    \end{align*}
    donde $C[0,  1]$ es el conjunto de funciones $f:[0,  1] \longrightarrow \RR$ continuas. Notemos que $T$ admite una interpretación geométrica sencilla, tomando una función arbitraria en $C[0,  1]$ como se muestra en la figura \ref{IDIDIIDKDKDKKK}. Así pues, comprobemos que $T$ es una transformación lineal:\sideFigure[\label{IDIDIIDKDKDKKK}Operador integral]{
    \begin{center}
        \begin{tikzpicture}[declare function={
        a=0;
        b=3;
        f(\x)=exp(\x) - \x + 1;
        }]
            \begin{axis}[
            axis lines=middle,
            xlabel=$x$,
            ylabel=$y$,
            xmin=-0.5,
            xmax=3,
            ymin=-0.75,
            ymax=8,
            ytick=\empty,
            xtick={2},
            xticklabels={$1$},
            width=6.3cm,
            height=10cm,
            xlabel style={
                anchor=west,
            },
            ylabel style={
                anchor=south,
            },
            axis line style={thick,-Stealth},
            ]
                \addplot[name path=A, gray, thick, domain=0.01:2, smooth] {f(x)};
                \path[name path=B] (\pgfkeysvalueof{/pgfplots/xmin},0) -- (\pgfkeysvalueof{/pgfplots/xmax},0);
                \addplot[gray!20] fill between [of=A and B, soft clip={domain=a:2},];
                \addplot[dashed, thick,->,>={}] coordinates {(2,0)(2,8)};
                \path ({(2+a)/2},{f((2+a)/2)/2}) node{$\displaystyle \int_0^1 f(x)$};
                \path (0,0) node[below left] {$0$};
                \addplot[thick,->,>={}] coordinates {(0,6)(0,0)(2.5,0)};
            \end{axis}
        \end{tikzpicture}
    \end{center}
    }
    \begin{enumerate}[label=\roman*)]
        \item Sea $f$, $g \in C[0,  1]$. Entonces
        \begin{align*}
            T(f + g) & = \int_0^1 (f + g)(x) dx \\
            & = \int_0^1 f(x) + g(x) dx \\
            & = \int_0^1 f(x) dx + \int_0^1 g(x) dx \\
            & = T_{f} + T_{g}
        \end{align*}
        Por tanto, $T(f + g) = T_{f} + T_{g}$.
        \item[ii)] Sea $f \in C[0,  1]$ y $\alpha \in \RR$. Entonces
        \begin{align*}
            T(\alpha f) & = \int_0^1 (\alpha f)(x) dx \\
            & = \int_0^1 \alpha f(x) dx \\
            & = \alpha \int_0^1 f(x) dx \\
            & = \alpha T_{f}
        \end{align*}
        Por tanto, $T(\alpha f) = \alpha T_{f}$.
    \end{enumerate}
    Por tanto, $T$ es una transformación lineal.
\end{example}

\begin{example}
    Sea
    \begin{align*}
        T:C[0,  1] & \longrightarrow \RR \\
        f & \longmapsto T_{f} = f(0) + 1
    \end{align*}
    donde $C[0,  1]$ es el conjunto de funciones $f:[0,  1] \longrightarrow \RR$ continuas. Notemos que $T$ es no lineal, pues tenemos que
    $$T(f + g) = (f + g) + 1 = f(0) + g(0) + 1$$
    y
    $$T_{f} + T_{g} = [f(0) + 1] + [g(0) + 1] = f(0) + g(0) + 2$$
\end{example}

\begin{theorem}
    Sean $V$ y $W$ espacios vectoriales sobre $K$. Si $T: V \longrightarrow W$ es una transformación lineal, entonces
    \begin{enumerate}[label=\roman*)]
        \item $T\mathbb{0}_V = \mathbb{0}_W$.
        \item $T(-\mathbb{u}) = -T\mathbb{u}$.
        \item $T(\mathbb{u} - \mathbb{v}) = T\mathbb{u} - T\mathbb{v}$.
        \item $T(\alpha_1 \mathbb{u}_1 + \alpha_2 \mathbb{u}_2 + \cdots + \alpha_n \mathbb{u}_n) = \alpha_1 T\mathbb{u}_1 + \alpha_2 T\mathbb{u}_2 + \cdots + \alpha_n T\mathbb{u}_n$ con $\alpha_1$, $\alpha_2$, $\dots$, $\alpha_n \in K$ y $\mathbb{u}_1$, $\mathbb{u}_2$, $\dots$, $\mathbb{u}_n \in V$.
    \end{enumerate}
    \demostracion
    \begin{enumerate}[label=\roman*)]
        \item Tenemos que
        $$\mathbb{0}_V = \mathbb{0}_V + \mathbb{0}_V$$
        así que
        \begin{align*}
            T\mathbb{0}_V & = T(\mathbb{0}_V + \mathbb{0}_V) \\
            & = T\mathbb{0}_V + T\mathbb{0}_V
        \end{align*}
        Ahora
        \begin{align*}
            \mathbb{0}_W & = T\mathbb{0}_V + (-T\mathbb{0}_V) \\
            & = T\mathbb{0}_V + T\mathbb{0}_V + (-T\mathbb{0}_V) \\
            & = T\mathbb{0}_V + \big( T\mathbb{0}_V + (-T\mathbb{0}_V) \big) \\
            & = T\mathbb{0}_V + \mathbb{0}_W \\
            & = T\mathbb{0}_V
        \end{align*}
        Por tanto, $T\mathbb{0}_V = \mathbb{0}_W$.
        \item Sea $\mathbb{u} \in V$, entonces
        \begin{align*}
            T\mathbb{u} + T(-\mathbb{u}) & = T\big( \mathbb{u} + (-\mathbb{u}) \big) \\
            & = T(\mathbb{0}_V) \\
            & = \mathbb{0}_W
        \end{align*}
        Por tanto, $T(-\mathbb{u}) = -T\mathbb{u}$.
        \item Sean $\mathbb{u}$, $\mathbb{v} \in V$, entonces
        \begin{align*}
            T(\mathbb{u} - \mathbb{v}) & = T\mathbb{u} + T(-\mathbb{v}) \\
            & = T\mathbb{u} - T\mathbb{v}
        \end{align*}
        Por tanto, $T(\mathbb{u} - \mathbb{v}) = T\mathbb{u} - T\mathbb{v}$.
        \item Sean $\alpha_1$, $\alpha_2$, $\dots$, $\alpha_n \in K$ y sean $\mathbb{u}_1$, $\mathbb{u}_2$, $\dots$, $\mathbb{u}_n \in V$. Procedamos por inducción sobre $n$. Si $n = 2$, entonces
        \begin{align*}
            T(\alpha_1 \mathbb{u}_1 + \alpha_2 \mathbb{u}_2) & = T(\alpha_1 \mathbb{u}_1) + T(\alpha_2 \mathbb{u}_2) \\
            & = \alpha_1 T\mathbb{u}_1 + \alpha_2 T\mathbb{u}_2
        \end{align*}
        Supongamos que se cumple para $k$, es decir,
        $$T(\alpha_1 \mathbb{u}_1 + \alpha_2 \mathbb{u}_2 + \cdots + \alpha_k \mathbb{u}_k) = \alpha_1 T\mathbb{u}_1 + \alpha_2 T\mathbb{u}_2 + \cdots + \alpha_k T\mathbb{u}_k$$
        Probemos ahora para $k+1$,
        \begin{align*}
            T(\alpha_1 \mathbb{u}_1 + \alpha_2 \mathbb{u}_2 + \cdots + \alpha_k \mathbb{u}_k + \alpha_{k+1} \mathbb{u}_{k+1}) & = T(\alpha_1 \mathbb{u}_1 + \alpha_2 \mathbb{u}_2 + \cdots + \alpha_k \mathbb{u}_k) + T(\alpha_{k+1} \mathbb{u}_{k+1}) \\
            & = \alpha_1 T\mathbb{u}_1 + \alpha_2 T\mathbb{u}_2 + \cdots + \alpha_k T\mathbb{u}_k + \alpha_{k+1} T\mathbb{u}_{k+1}
        \end{align*}
        Por tanto,
        $$T(\alpha_1 \mathbb{u}_1 + \alpha_2 \mathbb{u}_2 + \cdots + \alpha_n \mathbb{u}_n) = \alpha_1 T\mathbb{u}_1 + \alpha_2 T\mathbb{u}_2 + \cdots + \alpha_n T\mathbb{u}_n$$
    \end{enumerate}
\end{theorem}

\begin{theorem}\label{theorem:LAKAAKLKSKSKSSIIS}
    Sean $V$ y $W$ dos espacios vectoriales sobre $K$ y sean $T_1$, $T_2:V \longrightarrow W$ dos transformaciones lineales tales que
    $$T_1(\mathbb{v}_i) = T_2(\mathbb{v}_i), \text{ para } i = 1,  2,  \dots,  n$$
    siendo $\{ \mathbb{v}_1,  \mathbb{v}_2,  \dots,  \mathbb{v}_n \}$ una base de $V$, entonces $T_1 = T_2$. \\
    \demostracion
    Sea $\{ \mathbb{v}_1,  \mathbb{v}_2,  \dots,  \mathbb{v}_n \}$ una base de $V$. Basta probar que
    $$T_1\mathbb{v} = T_2 \mathbb{v}, \forall \mathbb{v} \in V$$
    Así pues, sea $\mathbb{v} \in V$ un elemento arbitrario, entonces
    \begin{equation}
        \mathbb{v} = \alpha_1 \mathbb{v}_1 + \alpha_2 \mathbb{v}_2 + \cdots + \alpha_n \mathbb{v}_n \label{JAJJSJSKSKDKKSKDSKDKIDD}
    \end{equation}
    con $\alpha_i \in K$ para $i = 1$, $2$, $\dots$, $n$, ya que $\{ \mathbb{v}_1,  \mathbb{v}_2,  \dots,  \mathbb{v}_n \}$ es una base de $V$. De \eqref{JAJJSJSKSKDKKSKDSKDKIDD},
    \begin{align*}
        T_1 \mathbb{v} & = T_1(\alpha_1 \mathbb{v}_1 + \alpha_2 \mathbb{v}_2 + \cdots + \alpha_n \mathbb{v}_n) \\
        & = T_1(\alpha_1 \mathbb{v}_1) + T_1(\alpha_2 \mathbb{v}_2) + \cdots + T_1(\alpha_n \mathbb{v}_n) \\
        & = \alpha_1 T_1 \mathbb{v}_1 + \alpha_2 T_1 \mathbb{v}_2 + \cdots + \alpha_n T_1 \mathbb{v}_n \\
        & = \alpha_1 T_2 \mathbb{v}_1 + \alpha_2 T_2 \mathbb{v}_2 + \cdots + \alpha_n T_2 \mathbb{v}_n \\
        & = T_2(\alpha_1 \mathbb{v}_1) + T_2(\alpha_2 \mathbb{v}_2) + \cdots + T_2(\alpha_n \mathbb{v}_n) \\
        & = T_2(\alpha_1 \mathbb{v}_1 + \alpha_2 \mathbb{v}_2 + \cdots + \alpha_n \mathbb{v}_n) \\
        & = T_2 \mathbb{v}
    \end{align*}
    Por tanto, $T_1 \mathbb{v} = T_2 \mathbb{v}$, $\forall \mathbb{v} \in V$. Así, $T_1 = T_2$.
\end{theorem}

\section{Núcleo e imagen de una transformación lineal}

\begin{definition}
    Sean $V$ y $W$ espacios vectoriales sobre $K$ y $T:V \longrightarrow W$ una transformación lineal. Se define:
    \begin{enumerate}[label=\roman*)]
        \item El núcleo de la transformación lineal $T$, como
        $$\Nuc(T) = \left\{ \mathbb{v} \in V \mid T\mathbb{v} = \mathbb{0}_W \right\}$$
        \item La imagen de la transformación lineal $T$, como
        $$\Ima(T) = \left\{ \mathbb{w} \in W \mid T\mathbb{v} = \mathbb{w}, \text{ para algún } \mathbb{v} \in V \right\}$$
        \item La nulidad de la transformación lineal $T$, como
        $$\nu(T) = \Dim \big( \Nuc(T) \big)$$
        \item El rango de la transformación lineal $T$, como
        $$\rho(T) = \Dim \big( \Ima(T) \big)$$
    \end{enumerate}
\end{definition}

\newpage

\begin{theorem}
    Sean $V$ y $W$ espacios vectoriales sobre $K$ y $T:V \longrightarrow W$ una transformación lineal, entonces $\Nuc(T)$ es subespacio de $V$ e $\Ima(T)$ es subespacio de $W$. \\
    \demostracion
    Primero, probemos que $\Nuc(T)$ es subespacio de $V$. Sean $\mathbb{v}_1$, $\mathbb{v}_2 \in \Nuc(T)$ y $\alpha \in K$, entonces
    \begin{equation}
        T\mathbb{v}_1 = \mathbb{0}_W \quad \text{ y } \quad T\mathbb{v}_2 = \mathbb{0}_W
    \end{equation}
    Así,
    \begin{align*}
        T(\mathbb{v}_1 + \mathbb{v}_2) & = T\mathbb{v}_1 + T\mathbb{v}_2 \\
        & = \mathbb{0}_W + \mathbb{0}_W \\
        & = \mathbb{0}_W
    \end{align*}
    Entonces
    \begin{equation}
        \mathbb{v}_1 + \mathbb{v}_2 \in \Nuc(T) \label{JAUDJKDKDKDKD}
    \end{equation}
    Además
    \begin{align*}
        T(\alpha \mathbb{v}_1) & = \alpha T\mathbb{v}_1 \\
        & = \alpha \mathbb{0}_W \\
        & = \mathbb{0}_W
    \end{align*}
    Entonces
    \begin{equation}
        \alpha \mathbb{v}_1 \in \Nuc(T) \label{OSPWPEOSLDDPDPD}
    \end{equation}
    Por tanto, de \eqref{JAUDJKDKDKDKD} y \eqref{OSPWPEOSLDDPDPD}, $\Nuc(T)$ es subespacio de $V$.

    Ahora veamos que $\Ima(T)$ es subespacio de $W$. Sean $\mathbb{w}_1$, $\mathbb{w}_2 \in \Ima(T)$ y $\alpha \in K$, entonces
    \begin{equation}
        \mathbb{w}_1 = T\mathbb{v}_1, \text{ para algún } \mathbb{v}_1 \in V \quad \text{ y } \quad \mathbb{w}_2 = T\mathbb{v}_2, \text{ para algún } \mathbb{v}_2 \in V
    \end{equation}
    Así,
    \begin{align*}
        \mathbb{w}_1 + \mathbb{w}_2 & = T\mathbb{v}_1 + T\mathbb{v}_2 \\
        & = T(\mathbb{v}_1 + \mathbb{v}_2) \\
        & = T\mathbb{\mu}
    \end{align*}
    siendo $\mathbb{\mu} = \mathbb{v}_1 + \mathbb{v}_2$. Entonces
    \begin{equation}
        \mathbb{w}_1 + \mathbb{w}_2 \in \Ima(T) \label{IAISISKSKSKLS}
    \end{equation}
    Además
    \begin{align*}
        \alpha \mathbb{w}_1 & = \alpha T\mathbb{v}_1 \\
        & = T(\alpha \mathbb{v}_1) \\
        & = T(\mathbb{\upsilon})
    \end{align*}
    siendo $\mathbb{\upsilon} = \alpha \mathbb{v}_1$. Entonces
    \begin{equation}
        \alpha \mathbb{w}_1 \in \Ima(T) \label{ISOSPPSOSPASHSHSJ}
    \end{equation}
    Por tanto, de \eqref{IAISISKSKSKLS} y \eqref{ISOSPPSOSPASHSHSJ} $\Ima(T)$ es subespacio de $W$.
\end{theorem}

\begin{example}
    Dada la transformación lineal $T:\RR[4] \longrightarrow \RR[2]$ definida por
    $$T\begin{pmatrix}
        x \\
        y \\
        z \\
        w
    \end{pmatrix} = \begin{pmatrix}
        x + z \\
        y + w
    \end{pmatrix}$$
    Determine $\Nuc(T)$, $\Ima(T)$, $\nu(T)$ y $\rho(T)$. \newpage
    \solucion Por definición,
    \begin{align*}
        \Nuc(T) & = \left\{ \begin{pmatrix}
            x \\
            y \\
            z \\
            w
        \end{pmatrix} \in \RR[4] \mid T \begin{pmatrix}
            x \\
            y \\
            z \\
            w
        \end{pmatrix} = \begin{pmatrix}
            0 \\
            0
        \end{pmatrix} \right\} \\
        & = \left\{ \begin{pmatrix}
            x \\
            y \\
            z \\
            w
        \end{pmatrix} \in \RR[4] \mid \begin{pmatrix}
            x + z \\
            y + w
        \end{pmatrix} = \begin{pmatrix}
            0 \\
            0
        \end{pmatrix} \right\} \\
        & = \left\{ \begin{pmatrix}
            x \\
            y \\
            z \\
            w
        \end{pmatrix} \in \RR[4] \mid x + z = 0, \; y + w = 0 \right\} \\
        & = \left\{ \begin{pmatrix*}[r]
            x \\
            y \\
            -x \\
            -y
        \end{pmatrix*} \in \RR[4] \mid x,  y \in \RR \right\} \\
        & = \left\{ x \begin{pmatrix*}[r]
            1 \\
            0 \\
            -1 \\
            0
        \end{pmatrix*} + y \begin{pmatrix*}[r]
            0 \\
            1 \\
            0 \\
            -1
        \end{pmatrix*} \mid x,  y \in \RR \right\}
    \end{align*}
    Por tanto,
    $$\Nuc(T) = \Gen \left( \left\{ \begin{pmatrix*}[r]
        1 \\
        0 \\
        -1 \\
        0
    \end{pmatrix*},  \begin{pmatrix*}[r]
        0 \\
        1 \\
        0 \\
        -1
    \end{pmatrix*} \right\} \right)$$
    y como los vectores son l.i, entonces $\nu(T) = 2$. Ahora, por definición,
    \begin{align*}
        \Ima(T) & = \left\{ \begin{pmatrix}
            \gamma_1 \\
            \gamma_2
        \end{pmatrix} \in \RR[2] \mid T\begin{pmatrix}
            x \\
            y \\
            z \\
            w
        \end{pmatrix} = \begin{pmatrix}
            \gamma_1 \\
            \gamma_2
        \end{pmatrix}, \text{ para algún } \begin{pmatrix}
            x \\
            y \\
            z \\
            w
        \end{pmatrix} \in \RR[4] \right\} \\
        & = \left\{ \begin{pmatrix}
            \gamma_1 \\
            \gamma_2
        \end{pmatrix} \in \RR[2] \mid \begin{pmatrix}
            x + z \\
            y + w
        \end{pmatrix} = \begin{pmatrix}
            \gamma_1 \\
            \gamma_2
        \end{pmatrix}, \text{ para algún } \begin{pmatrix}
            x \\
            y \\
            z \\
            w
        \end{pmatrix} \in \RR[4] \right\} \\
        & = \left\{ \begin{pmatrix}
            \gamma_1 \\
            \gamma_2
        \end{pmatrix} = \begin{pmatrix}
            x + z \\
            y + w
        \end{pmatrix}, \text{ para algún } \begin{pmatrix}
            x \\
            y \\
            z \\
            w
        \end{pmatrix} \in \RR[4] \right\} \\
        & = \left\{ \begin{pmatrix}
            \gamma_1 \\
            \gamma_2
        \end{pmatrix} = x \begin{pmatrix}
            1 \\
            0
        \end{pmatrix} + y \begin{pmatrix}
            0 \\
            1
        \end{pmatrix} + z \begin{pmatrix}
            1 \\
            0
        \end{pmatrix} + w \begin{pmatrix}
            0 \\
            1
        \end{pmatrix}, \text{ para algún } x,  y,  z,  w \in \RR \right\} \\
        & = \left\{ \begin{pmatrix}
            \gamma_1 \\
            \gamma_2
        \end{pmatrix} = x \begin{pmatrix}
            1 \\
            0
        \end{pmatrix} + y \begin{pmatrix}
            0 \\
            1
        \end{pmatrix} + z \begin{pmatrix}
            1 \\
            0
        \end{pmatrix} + w \begin{pmatrix}
            0 \\
            1
        \end{pmatrix}, \text{ para algún } x,  y,  z,  w \in \RR \right\}
    \end{align*}
    Por tanto,
    $$\Ima(T) = \Gen \left( \left\{  \begin{pmatrix}
        1 \\
        0
    \end{pmatrix},  \begin{pmatrix}
        0 \\
        1
    \end{pmatrix},  \begin{pmatrix}
        1 \\
        0
    \end{pmatrix},  \begin{pmatrix}
        0 \\
        1
    \end{pmatrix} \right\} \right)$$
    y en este caso, vemos que solo dos vectores son l.i, así que $\rho(T) = 2$.
\end{example}

\begin{example}
    Sea $T: P_3(x) \longrightarrow P_2(x)$ una transformación lineal definida por
    $$Tp(x) = T \left( a_3x^3 + a_2x^2 + a_1x + a_0 \right) = a_2x^2 + a_1x + a_0$$
    Determine $\Nuc(T)$, $\Ima(T)$, $\nu(T)$ y $\rho(T)$. \newpage
    \solucion Por definición,
    \begin{align*}
        \Nuc(T) & = \left\{ p(x) \in P_3(x) \mid Tp(x) = 0 \right\} \\
        & = \left\{ p(x) \in P_3(x) \mid T \left( a_3x^3 + a_2x^2 + a_1x + a_0 \right) = 0 \right\} \\
        & = \left\{ p(x) \in P_3(x) \mid a_2x^2 + a_1x + a_0 = 0 \right\} \\
        & = \left\{ p(x) \in P_3(x) \mid a_0 = 0,  a_1 = 0,  a_2 = 0 \right\} \\
        & = \left\{ a_3x^3 + 0 \cdot x^2 + 0 \cdot x + 0 \cdot 0 \mid a_3 \in \RR \right\} \\
        & = \left\{ a_3x^3 \mid a_3 \in \RR \right\} \\
        & = \Gen \left( \left\{ x^3 \right\} \right)
    \end{align*}
    Por tanto,
    $$\Nuc(T) = \Gen \left( \left\{ x^3 \right\} \right)$$
    y por ser l.i, se sigue que $\nu(T) = 1$. Ahora, por definición,
    \begin{align*}
        \Ima(T) & = \left\{ q(x) \in P_2(x) \mid Tp(x) = q(x), \text{ para algún } p(x) \in P_3(x) \right\} \\
        & = \left\{ q(x) \in P_2(x) \mid a_2x^2 + a_1x + a_0 = q(x), \text{ para algún } p(x) \in P_3(x) \right\} \\
        & = \left\{ q(x) = a_2x^2 + a_1x + a_0 \mid a_2,  a_1,  a_0 \in \RR \right\} \\
        & = \left\{ a_2x^2 + a_1x + a_0 \mid a_2,  a_1,  a_0 \in \RR \right\} \\
        & = \Gen \left( \left\{ x^2,  x,  1 \right\} \right)
    \end{align*}
    Por tanto,
    $$\Ima(T) = \Gen \left( \left\{ x^2,  x,  1 \right\} \right)$$
    y como son l.i, entonces $\rho(T) = 3$.
\end{example}

\section{Representación matricial de una transformación lineal}

\begin{theorem}
    Sea $T: \RR[n] \longrightarrow \RR[m]$ una transformación lineal, entonces existe una única matriz $A_T \in \matrizmn$ tal que
    $$T\mathbb{x} = A_T\mathbb{x}, \; \forall \mathbb{x} \in \RR[n]$$
    A la matriz $A_T$, se le llama la representación matricial de la transformación lineal $T$. \\
    \demostracion Sea $\displaystyle \left\{ e_1 = \left( \begin{array}{c} 1 \\ 0 \\ \vdots \\ 0 \\ 0 \end{array} \right),  e_2 = \left( \begin{array}{c} 0 \\ 1 \\ \vdots \\ 0 \\ 0 \end{array} \right),  \dots,  e_n = \left( \begin{array}{c} 0 \\ 0 \\ \vdots \\ 0 \\ 1 \end{array} \right) \right\}$ la base canónica de $\RR[n]$. Sea
    $$Te_1 = \mathbb{w}_1 \in \RR[m],  Te_2 = \mathbb{w}_2 \in \RR[m],  \dots,  Te_n = \mathbb{w}_n \in \RR[m]$$
    siendo $\mathbb{w}_j = \begin{bmatrix}
        a_{1j} \\
        a_{2j} \\
        \vdots \\
        a_{mj}
    \end{bmatrix}$, para $j = 1,  2,  \dots,  n$. Sea $A_T = \begin{bmatrix}
        \mathbb{w}_1 & \mathbb{w}_2 & \cdots & \mathbb{w}_n
    \end{bmatrix}$ donde
    $$A_T = \begin{bmatrix}
        a_{11} & a_{12} & \cdots & a_{1n} \\
        a_{21} & a_{22} & \cdots & a_{2n} \\
        \vdots & & \ddots & \\
        a_{m1} & a_{m2} & \cdots & a_{mn}
    \end{bmatrix}$$\newpage\noindent
    Veamos lo siguiente
    $$Te_1 = \mathbb{w}_1 = \begin{bmatrix}
        a_{11} \\
        a_{21} \\
        \vdots \\
        a_{m1}
    \end{bmatrix} \quad \text{ y } \quad A_Te_1 = \begin{bmatrix}
        a_{11} \\
        a_{21} \\
        \vdots \\
        a_{m1}
    \end{bmatrix}$$
    entonces $Te_1 = A_Te_1$. Asimismo,
    $$Te_2 = \mathbb{w}_2 = \begin{bmatrix}
        a_{12} \\
        a_{22} \\
        \vdots \\
        a_{m2}
    \end{bmatrix} \quad \text{ y } \quad A_Te_2 = \begin{bmatrix}
        a_{12} \\
        a_{22} \\
        \vdots \\
        a_{m2}
    \end{bmatrix}$$
    entonces $Te_2 = A_Te_2$. Así pues,
    $$Te_n = \mathbb{w}_n = \begin{bmatrix}
        a_{1n} \\
        a_{2n} \\
        \vdots \\
        a_{mn}
    \end{bmatrix} \quad \text{ y } \quad A_Te_2 = \begin{bmatrix}
        a_{1n} \\
        a_{2n} \\
        \vdots \\
        a_{mn}
    \end{bmatrix}$$
    entonces $Te_n = A_Te_n$. Por el teorema \ref{theorem:LAKAAKLKSKSKSSIIS}, $T = A_T$.

    Ahora, veamos que $A_T$ es único. Supongamos que existe $B_T \in \matrizmn$ tal que
    $$T\mathbb{x} = B_T\mathbb{x}, \; \forall \mathbb{x} \in \RR[n]$$
    Así
    $$A_T\mathbb{x} - B_T\mathbb{x} = \mathbb{0}_{\RR[m]}$$
    de donde se sigue que
    $$A_T\mathbb{x} - B_T\mathbb{x} + B_T\mathbb{x} = \mathbb{0}_{\RR[m]} + B_T\mathbb{x}$$
    entonces
    $$A_T\mathbb{x} = B_T\mathbb{x}, \; \forall \mathbb{x} \in \RR[n]$$
    Por lo tanto, $A_T = B_T$, de donde se sigue que $A_T$ es única.
\end{theorem}

\begin{example}
    Sea $T:\RR[3] \longrightarrow \RR[4]$ definida por $T \begin{pmatrix}
        x \\
        y \\
        z
    \end{pmatrix} = \begin{bmatrix}
        x - y \\
        y + z \\
        2x - y - z \\
        - x + y + 2z
    \end{bmatrix}$. Determine $A_T$, $\Nuc(T)$, $\Ima(T)$, $\nu(T)$ y $\rho(T)$. \\
    \solucion Sea $\left\{ e_1 = \begin{pmatrix}
        1 \\
        0 \\
        0
    \end{pmatrix},  e_2 = \begin{pmatrix}
        0 \\
        1 \\
        0
    \end{pmatrix},  e_3 = \begin{pmatrix}
        0 \\
        0 \\
        1
    \end{pmatrix} \right\}$ la base canónica de $\RR[3]$. Ahora, por el teorema anterior,
    $$Te_1 = T\begin{pmatrix}
        1 \\
        0 \\
        0
    \end{pmatrix} = \begin{bmatrix*}[r]
        1 \\
        0 \\
        2 \\
        -1
    \end{bmatrix*} = \mathbb{w}_1, \quad Te_2 = T\begin{pmatrix}
        0 \\
        1 \\
        0
    \end{pmatrix} = \begin{bmatrix*}[r]
        -1 \\
        1 \\
        -1 \\
        1
    \end{bmatrix*} = \mathbb{w}_2, \quad Te_3 = T\begin{pmatrix}
        0 \\
        0 \\
        1
    \end{pmatrix} = \begin{bmatrix*}[r]
        0 \\
        1 \\
        -1 \\
        2
    \end{bmatrix*} = \mathbb{w}_3$$
    De esta forma,
    \begin{align*}
        A_T & = \begin{bmatrix}
            \mathbb{w}_1 & \mathbb{w}_2 & \mathbb{w}_3
        \end{bmatrix} \\
        & = \begin{bmatrix*}[r]
            1 & -1 & 0 \\
            0 & 1 & 1 \\
            2 & -1 & -1 \\
            -1 & 1 & 2
        \end{bmatrix*}
    \end{align*}
    es la representación matricial de $T$. Veamos lo siguiente: Para todo $\mathbb{x} \in \RR[3]$,
    \begin{align*}
        A_T\mathbb{x} & = \begin{bmatrix*}[r]
            1 & -1 & 0 \\
            0 & 1 & 1 \\
            2 & -1 & -1 \\
            -1 & 1 & 2
        \end{bmatrix*} \begin{pmatrix}
            x \\
            y \\
            z
        \end{pmatrix} \\
        & = \begin{bmatrix}
            x - y \\
            y + z \\
            2x - y - z \\
            - x + y + 2z
        \end{bmatrix} \\
        & = T \begin{pmatrix}
            x \\
            y \\
            z
        \end{pmatrix}
    \end{align*}
    Ahora,
    \begin{align*}
        \Nuc(T) & = \left\{ \mathbb{x} \in \RR[3] \mid T\mathbb{x} = \mathbb{0} \right\} \\
        & = \left\{ \begin{pmatrix}
            x \\
            y \\
            z
        \end{pmatrix} \in \RR[3] \mid \begin{bmatrix}
            x - y \\
            y + z \\
            2x - y - z \\
            - x + y + 2z
        \end{bmatrix} = \begin{pmatrix}
            0 \\
            0 \\
            0 \\
            0
        \end{pmatrix} \right\}
    \end{align*}
    entonces
    \begin{align*}
        x - y & = 0 \\
        y + z & = 0 \\
        2x - y - z & = 0 \\
        - x + y + 2z & = 0
    \end{align*}
    de donde $x = y = z = 0$. Por lo que $\Nuc(T) = \left\{ \begin{pmatrix}
        0 \\
        0 \\
        0
    \end{pmatrix} \right\}$, entonces $\nu(T) = 0$. Finalmente,
    \begin{align*}
        \Ima(T) & = \left\{ \mathbb{y} \in \RR[4] \mid T\mathbb{x} = \mathbb{y}, \text{ para algún } \mathbb{x} \in \RR[3] \right\} \\
        & = \left\{ \mathbb{y} \in \RR[4] \mid \begin{bmatrix}
            x - y \\
            y + z \\
            2x - y - z \\
            - x + y + 2z
        \end{bmatrix} = \mathbb{y}, \text{ para algún } \begin{pmatrix}
            x \\
            y \\
            z
        \end{pmatrix} \in \RR[3] \right\} \\
        & = \left\{ x \begin{pmatrix*}[r]
            1 \\
            0 \\
            2 \\
            -1
        \end{pmatrix*} + y \begin{pmatrix*}[r]
            -1 \\
            1 \\
            -1 \\
            1
        \end{pmatrix*} + z \begin{pmatrix*}[r]
            0 \\
            1 \\
            -1 \\
            2
        \end{pmatrix*} \mid x,  y,  z \in \RR \right\}
    \end{align*}
    Por tanto,
    $$\Ima(T) = \Gen \left( \left\{ \begin{pmatrix*}[r]
        1 \\
        0 \\
        2 \\
        -1
    \end{pmatrix*},  \begin{pmatrix*}[r]
        -1 \\
        1 \\
        -1 \\
        1
    \end{pmatrix*},  \begin{pmatrix*}[r]
        0 \\
        1 \\
        -1 \\
        2
    \end{pmatrix*} \right\} \right)$$
    y como son l.i, se sigue que $\rho(T) = 3$.
\end{example}

\begin{theorem}
    Dada $T:\RR[n] \longrightarrow \RR[m]$ y $A_T \in \matrizmn$ su representación matricial, entonces
    \begin{enumerate}[label=\roman*)]
        \item $\Nuc(T) = \Nuc(A_T)$.
        \item $\Ima(T) = \Ima(A_T)$.
        \item $\nu(T) = \nu(A_T)$.
        \item $\rho(T) = \rho(A_T)$.
    \end{enumerate}
\end{theorem}

\newpage

\section{Isomorfismos}

\begin{definition}
    Sea $T:V \longrightarrow W$ una transformación lineal. Decimos que $T$ es uno a uno (o inyectiva) si
    $$T\mathbb{u} = T\mathbb{v} \Longrightarrow \mathbb{u} = \mathbb{v}$$
    siendo $\mathbb{u}$, $\mathbb{v} \in V$. Equivalentemente, $T$ es uno a uno si
    $$T\mathbb{u} \neq T\mathbb{v} \Longrightarrow \mathbb{u} \neq \mathbb{v}$$
    siendo $\mathbb{u}$, $\mathbb{v} \in V$.
\end{definition}

\begin{theorem}\label{theo:nu_cero}
    Sea $T:V \longrightarrow W$ una transformación lineal, $T$ es uno a uno si y solo si
    $$\Nuc(T) = \{ \mathbb{0}_V \}$$
    \demostracion
    \begin{enumerate}
        \item[$\bm{\Rightarrow}$)] Supongamos que $T$ es uno a uno, hay que demostrar que
        $$\Nuc(T) = \{ \mathbb{0}_V \}$$
        Como $T$ es uno a uno, entonces
        \begin{equation}
            T\mathbb{u} = T\mathbb{v} \Longrightarrow \mathbb{u} = \mathbb{v}, \text{ para } \mathbb{u},  \mathbb{v} \in V \label{JAJJANNJBAJBABBGTQQH}
        \end{equation}
        De \eqref{JAJJANNJBAJBABBGTQQH},
        \begin{align*}
            \mathbb{0}_W & = T\mathbb{v} - T\mathbb{v} \\
            & = T\mathbb{u} - T\mathbb{v} \\
            & = T(\mathbb{u} - \mathbb{v})
        \end{align*}
        Por lo que $T(\mathbb{u} - \mathbb{v}) = \mathbb{0}_W$ de donde se sigue que $\mathbb{u} - \mathbb{v} \in \Nuc(T)$. Sea $\mathbb{v} \in \Nuc(T)$, entonces
        \begin{equation}
            T\mathbb{v} = \mathbb{0}_W \label{JAIAIJAKJJBJBA}
        \end{equation}
        Además,
        \begin{equation}
            T\mathbb{0}_V = \mathbb{0}_W \label{JAINNJJJAJJIA}
        \end{equation}
        De \eqref{JAIAIJAKJJBJBA} y \eqref{JAINNJJJAJJIA},
        $$T\mathbb{0}_V = T\mathbb{v}$$
        entonces $\mathbb{v} = \mathbb{0}$ por ser $T$ uno a uno. Por tanto, $\Nuc(T) = \{ \mathbb{0} \}$.
        \item[$\bm{\Leftarrow}$)] Supongamos que $\Nuc(T) = \{ \mathbb{0} \}$, hay que demostrar que $T$ es uno a uno. Sea
        $$T\mathbb{u} = T\mathbb{v}, \text{ con } \mathbb{u},  \mathbb{v} \in V$$
        Entonces
        \begin{align*}
            \mathbb{0}_W & = T\mathbb{v} - T\mathbb{v} \\
            & = T\mathbb{u} - T\mathbb{v} \\
            & = T(\mathbb{u} - \mathbb{v})
        \end{align*}
        entonces $\mathbb{u} - \mathbb{v} \in \Nuc(T) = \{ \mathbb{0} \}$. Así, $\mathbb{u} - \mathbb{v} = \mathbb{0}$; por lo tanto, $\mathbb{u} = \mathbb{v}$, lo que muestra que $T$ es uno a uno.
    \end{enumerate}
\end{theorem}

\begin{example}
    Verifique que la transformación lineal $T:\RR[2] \longrightarrow \RR[2]$, definida como
    $$T \begin{pmatrix}
        x \\
        y
    \end{pmatrix} = \begin{pmatrix}
        x + y \\
        x - y
    \end{pmatrix}$$
    es uno a uno. \newpage
    \solucion Por definición,
    \begin{align*}
        \Nuc(T) & = \left\{ \begin{pmatrix}
            x \\
            y
        \end{pmatrix} \in \RR[2] \mid T\begin{pmatrix}
            x \\
            y
        \end{pmatrix} = \begin{pmatrix}
            0 \\
            0
        \end{pmatrix} \right\} \\
        & = \left\{ \begin{pmatrix}
            x \\
            y
        \end{pmatrix} \in \RR[2] \mid \begin{pmatrix}
            x + y \\
            x - y
        \end{pmatrix} = \begin{pmatrix}
            0 \\
            0
        \end{pmatrix} \right\} \\
        & = \left\{ \begin{pmatrix}
            x \\
            y
        \end{pmatrix} \in \RR[2] \mid x + y = 0,  x - y = 0 \right\} \\
        & = \left\{ \begin{pmatrix}
            x \\
            y
        \end{pmatrix} \in \RR[2] \mid -x = x,  y = x \right\} \\
        & = \left\{ \begin{pmatrix}
            x \\
            y
        \end{pmatrix} \in \RR[2] \mid x = 0,  y = 0 \right\}
    \end{align*}
    Por tanto, $\displaystyle \Nuc(T) = \left\{ \begin{pmatrix}
        0 \\
        0
    \end{pmatrix} \right\}$, y por el teorema anterior, $T$ es uno a uno.
\end{example}

\begin{example}
    Verifique que la transformación lineal $T:\RR[2] \longrightarrow \RR[2]$, definida como
    $$T \begin{pmatrix}
        x \\
        y
    \end{pmatrix} = \begin{pmatrix}
        x + y \\
        x + y
    \end{pmatrix}$$
    no es uno a uno. \\
    \solucion Por definición,
    \begin{align*}
        \Nuc(T) & = \left\{ \begin{pmatrix}
            x \\
            y
        \end{pmatrix} \in \RR[2] \mid T\begin{pmatrix}
            x \\
            y
        \end{pmatrix} = \begin{pmatrix}
            0 \\
            0
        \end{pmatrix} \right\} \\
        & = \left\{ \begin{pmatrix}
            x \\
            y
        \end{pmatrix} \in \RR[2] \mid \begin{pmatrix}
            x + y \\
            y + y
        \end{pmatrix} = \begin{pmatrix}
            0 \\
            0
        \end{pmatrix} \right\} \\
        & = \left\{ \begin{pmatrix}
            x \\
            y
        \end{pmatrix} \in \RR[2] \mid x + y = 0,  x + y = 0 \right\} \\
        & = \left\{ \begin{pmatrix}
            x \\
            y
        \end{pmatrix} \in \RR[2] \mid y = - x \right\} \\
        & = \left\{ \begin{pmatrix*}[r]
            x \\
            -x
        \end{pmatrix*} \mid x \in \RR \right\} 
    \end{align*}
    Por tanto, $\displaystyle \Nuc(T) = \Gen \left( \left\{ \begin{pmatrix*}[r]
        1 \\
        -1
    \end{pmatrix*} \right\} \right) \neq \begin{pmatrix}
        0 \\
        0
    \end{pmatrix}$, y por el teorema anterior, $T$ no es uno a uno.
\end{example}

\begin{definition}
    Sea $T:V \longrightarrow W$ una transformación lineal. Decimos que $T$ es sobre si para todo $\mathbb{w} \in W$, existe al menos $\mathbb{v} \in V$ tal que $T\mathbb{v} = \mathbb{w}$.
\end{definition}

\begin{observation}
    Una transformación lineal $T:V \longrightarrow W$ sea uno a uno o sobre, admite una interpretación como se muestra en la siguiente figura:
    \begin{figure}[h!]
    \centering
    \begin{minipage}[c]{0.4\textwidth}
        \begin{center}
            \begin{tikzpicture}
                \foreach[count=\i] \lseti/\lsetmi in {V/{$\mathbb{u}$,$\mathbb{v}$,$\mathbb{w}$},W/{$T\mathbb{u}$,$T\mathbb{v}$,$T\mathbb{w}$}} {
                \begin{scope}[local bounding box=\lseti, x=3cm, y=0.5cm]
                    \foreach[count=\j] \lj in \lsetmi {
                    \node[minimum width=1em] (n-\j-\lseti) at (\i,-\j) {\lj};
                    }
                \end{scope}
                \node[ellipse = (1.5cm and 3cm), draw=black, thick, fit=(\lseti), label={[name=l-\lseti]above:$\lseti$}] {};
                }
                \draw[-latex] (n-1-V) -- (n-1-W);
                \draw[-latex] (n-2-V) -- (n-2-W);
                \draw[-latex] (n-3-V) -- (n-3-W);
                \draw[-latex] (l-V) -- node[above]{$T$}(l-V.center-|l-W.west);
            \end{tikzpicture}
            
            \textbf{(a)} Transformación lineal uno a uno
        \end{center}
    \end{minipage} \hspace{0.5cm}
    \begin{minipage}[c]{0.4\textwidth}
        \begin{center}
            \begin{tikzpicture}
                \foreach[count=\i] \lseti/\lsetmi in {V/{$\mathbb{u}$,$\mathbb{v}$,$\mathbb{w}$},W/{$T\mathbb{u}$,$T\mathbb{v}$,$T\mathbb{w}$}} {
                \begin{scope}[local bounding box=\lseti, x=3cm, y=0.5cm]
                    \foreach[count=\j] \lj in \lsetmi {
                    \node[minimum width=1em] (n-\j-\lseti) at (\i,-\j) {\lj};
                    }
                \end{scope}
                \node[ellipse = (1.5cm and 3cm), draw=black, thick, fit=(\lseti), label={[name=l-\lseti]above:$\lseti$}] {};
                }
                \draw[-latex] (n-1-V) -- (n-1-W);
                \draw[-latex, dash pattern=on 3pt off 3pt] (n-2-V) -- (n-1-W);
                \draw[-latex, dash pattern=on 3pt off 3pt] (n-3-V) -- (n-1-W);
                \draw[-latex] (l-V) -- node[above]{$T$}(l-V.center-|l-W.west);
            \end{tikzpicture}
            
            \textbf{(b)} Transformación lineal sobre
        \end{center}
    \end{minipage}
    \caption{Interpretación de una transformación uno a uno y sobre}
    \end{figure}
\end{observation}

\newpage

\begin{observation}
    $T$ es sobre si y solo si $\Ima(T) = W$.
\end{observation}

\begin{definition}
    Sea $T:V \longrightarrow W$ una transformación lineal. Decimos que $T$ es un isomorfismo si $T$ es uno a uno y sobre, y decimos que $V$ y $W$ son espacios vectoriales isomorfos y se denota como $V \cong W$.
\end{definition}

\begin{theorem}\label{theo:unoauno-sobre}
    Sea $T:V \longrightarrow W$ una transformación lineal con $\Dim V = n = \Dim W$.
    \begin{enumerate}[label=\roman*)]
        \item Si $T$ es uno a uno, entonces $T$ es sobre.
        \item Si $T$ es sobre, entonces $T$ es uno a uno.
    \end{enumerate}
    \demostracion
    \begin{enumerate}[label=\roman*)]
        \item Sea $T$ uno a uno. Como $\Dim(V) = n$, entonces $\{ \mathbb{v}_1,  \mathbb{v}_2,  \dots,  \mathbb{v}_n \}$ es una base de $V$. Ahora
        $$T\mathbb{v}_i = \mathbb{w}_i, \text{ para } i = 1, 2, \dots, n$$
        Además, $\mathbb{w}_i = T\mathbb{v}_i \neq T\mathbb{v}_j = \mathbb{w}_j$, donde se sigue que $\mathbb{v}_i \neq \mathbb{v}_j$, puesto que $T$ es uno a uno. Ahora, $\{ \mathbb{w}_1, \mathbb{w}_2, \dots,  \mathbb{w}_n \}$ es una base de $W$, veamos que son l.i, es decir, para $b_i \in K$
        \begin{align*}
            \mathbb{0}_W & = b_1\mathbb{w}_1 + b_2\mathbb{w}_2 + \cdots + b_n \mathbb{w}_n \\
            & = b_1T\mathbb{v}_1 + b_2T\mathbb{v}_2 + \cdots + b_nT\mathbb{v}_n \\
            & = T(b_1\mathbb{v}_1) + T(b_2\mathbb{v}_2) + \cdots + T(b_n\mathbb{v}_n) \\
            & = T(b_1\mathbb{v}_1 + b_2\mathbb{v}_2 + \cdots + b_n\mathbb{v}_n)
        \end{align*}
        entonces $b_1\mathbb{v}_1 + b_2\mathbb{v}_2 + \cdots + b_n\mathbb{v}_n \in \Nuc(T)$. Como $T$ es uno a uno, por el teorema anterior, $\Nuc(T) = \{ \mathbb{0}_V \}$; por lo que
        $$\mathbb{0}_V = b_1\mathbb{v}_1 + b_2\mathbb{v}_2 + \cdots + b_n\mathbb{v}_n$$
        entonces $b_1 = 0$, $b_2 = 0$, $\dots$, $b_n = 0$ ya que $\{ \mathbb{v}_1,  \mathbb{v}_2,  \dots,  \mathbb{v}_n \}$ es una base de $V$. Por tanto, $\mathbb{w}_1,  \mathbb{w}_2,  \dots,  \mathbb{w}_n$ son l.i. Sea $\mathbb{w} \in W$ arbitraria, entonces
        \begin{equation}
            \mathbb{w} = \alpha_1 \mathbb{w}_1 + \alpha_2 \mathbb{w}_2 + \cdots + \alpha_n \mathbb{w}_n \label{JHDFHGBSDHFGHDSGFHGH}
        \end{equation}
        siendo $\alpha_i \in K$ para $i = 1,  2,  \dots,  n$ y $\{ \mathbb{w}_1,  \mathbb{w}_2,  \dots, \mathbb{w}_n \}$ una base de $W$. Como $T\mathbb{v}_i = \mathbb{w}_i$, para $i = 1,  2,  \dots,  n$, entonces de \eqref{JHDFHGBSDHFGHDSGFHGH} se sigue que
        \begin{align*}
            \mathbb{w} & = \alpha_1T\mathbb{v}_1 + \alpha_2T\mathbb{v}_2 + \cdots + \alpha_nT\mathbb{v}_n \\
            & = T(\alpha_1\mathbb{v}_1) + T(\alpha_2\mathbb{v}_2) + \cdots + T(\alpha_n\mathbb{v}_n) \\
            & = T(\alpha_1\mathbb{v}_1 + \alpha_2\mathbb{v}_2 + \cdots + \alpha_n\mathbb{v}_n) \\
            & = T\mathbb{v}
        \end{align*}
        siendo
        $$\mathbb{v} = \alpha_1\mathbb{v}_1 + \alpha_2\mathbb{v}_2 + \cdots + \alpha_n\mathbb{v}_n \in V$$
        Esto es, dado $\mathbb{w} \in W$, existe al menos un $\mathbb{v} \in V$ tal que $T\mathbb{v} = \mathbb{w}$, es decir, $T$ es sobre.
        \item Se deja como ejercicio al lector.
    \end{enumerate}
    En conclusión, $V \cong W$.
\end{theorem}

\begin{definition}
    Sea $T:V \longrightarrow W$ una transformación lineal. Decimos que $\mathcal{T}:W \longrightarrow V$ es la transformación inversa de $T$ si $T \circ \mathcal{T} = I_V$ y se denota por $T^{-1}$, es decir, $\mathcal{T} = T^{-1}$.
\end{definition}

\newpage

\section{Cambio de base}

El cambio de base es un tema fundamental en álgebra lineal que se utiliza para relacionar las coordenadas de un espacio vectorial expresadas respecto a dos bases distintas. En otras palabras, el cambio de base permite transformar un vector de un espacio vectorial a otro espacio vectorial. La matriz del cambio de base es una herramienta matemática que se utiliza para transformar un vector de un espacio vectorial a otro espacio vectorial.

Para ilustrar el concepto de cambio de base, considere el siguiente ejemplo: Supongamos que tenemos un vector $\mathbb{v}$ expresado en términos de una base $\mathcal{B}_1$. Al cambiar a una nueva base $\mathcal{B}_2$, este vector se representaría mediante diferentes coordenadas, revelando cómo las bases afectan nuestra percepción y descripción de los objetos matemáticos. Para que nuestro ejemplo admita una representación geométrica, consideremos que
$$\mathcal{B}_1 = \left\{ e_1 = \begin{pmatrix}
    1 \\
    0
\end{pmatrix},  e_2 = \begin{pmatrix}
    0 \\
    1
\end{pmatrix} \right\} \quad \text{y} \quad \mathcal{B}_2 = \left\{ \mathbb{u}_1 = \begin{pmatrix}
    2 \\
    1
\end{pmatrix},  \mathbb{u}_2 = \begin{pmatrix}
    1 \\
    2
\end{pmatrix} \right\}$$

Estas dos bases se pueden representar de la siguiente manera:
\begin{figure}[h!]
    \centering
    \begin{tikzpicture}[scale=1.4]
        \draw[-Stealth, thick] (-1,0) -- (3,0);
        \draw[-Stealth, thick] (0,-1) -- (0,3);

        \draw[-latex, thick] (0,0) -- (1,0) node[below] {$e_1$};
        \draw[-latex, thick] (0,0) -- (0,1) node[left] {$e_2$};

        \draw[-latex, thick] (0,0) -- (2,1) node[right] {$\mathbb{u}_1$};
        \draw[-latex, thick] (0,0) -- (1,2) node[right] {$\mathbb{u}_2$};
    \end{tikzpicture}
    \caption{Representación geométrica de las bases $\mathcal{B}_1$ y $\mathcal{B}_2$}
\end{figure}

Deseamos encontrar algo que nos permita expresar cualquier vector de $\mathcal{B}_1$ en términos de $\mathcal{B}_2$. Así, buscamos dos escalares tales que
\begin{align*}
    e_1 & = a \mathbb{u}_1 + b \mathbb{u}_2 \\
    & = a \begin{pmatrix}
        2 \\
        1
    \end{pmatrix} + b \begin{pmatrix}
        1 \\
        2
    \end{pmatrix} \\
    & = \begin{pmatrix}
        2a + b \\
        a + 2b
    \end{pmatrix}
\end{align*}
entonces $a = 2/3$ y $b = -1/3$. Por tanto,
\begin{equation}
    e_1 = \frac{2}{3} \mathbb{u}_1 - \frac{1}{3} \mathbb{u}_2 \label{JAJAJAJAJHGVVAHAV}
\end{equation}
De manera análoga,
\begin{align*}
    e_2 & = \alpha \mathbb{u}_1 + \beta \mathbb{u}_2 \\
    & = \alpha \begin{pmatrix}
        2 \\
        1
    \end{pmatrix} + \beta \begin{pmatrix}
        1 \\
        2
    \end{pmatrix} \\
    & = \begin{pmatrix}
        2\alpha + \beta \\
        \alpha + 2\beta
    \end{pmatrix}
\end{align*}
entonces $\alpha = -1/3$ y $\beta = 2/3$. Por tanto,
\begin{equation}
    e_2 = -\frac{1}{3} \mathbb{u}_1 + \frac{2}{3} \mathbb{u}_2 \label{HABAVXCAHAHHAHU}
\end{equation}
Así, con \eqref{JAJAJAJAJHGVVAHAV} y \eqref{HABAVXCAHAHHAHU} podemos expresar cualquier vector de $\mathcal{B}_1$ en términos de $\mathcal{B}_2$. Por ejemplo, si tenemos $\mathbb{v} = \begin{pmatrix}
    4 \\
    3
\end{pmatrix}$ de la base $\mathcal{B}_1$, podemos expresarlo en términos de la base $\mathcal{B}_2$ como sigue:
\begin{align*}
    \mathbb{v} & = 4 \begin{pmatrix}
        1 \\
        0
    \end{pmatrix} + 3 \begin{pmatrix}
        0 \\
        1
    \end{pmatrix} \\
    & = 4e_1 + 3e_2 \\
    & = 4 \left[ \frac{2}{3} \mathbb{u}_1 - \frac{1}{3} \mathbb{u}_2 \right] + 3 \left[ -\frac{1}{3} \mathbb{u}_1 + \frac{2}{3} \mathbb{u}_2 \right] \\
    & = \frac{5}{3} \mathbb{u}_1 + \frac{2}{3} \mathbb{u}_2
\end{align*}

Observemos la tabla \ref{JAJAJAVGQGTQGQa}, de donde se obtiene\sideTable[\label{JAJAJAVGQGTQGQa}]{
\centering
\begin{tabular}{ccc}
    \toprule
    & $\mathbb{u}_1$ & $\mathbb{u}_2$ \\
    \midrule
    $e_1$ & $2/3$ & $-2/3$ \\
    $e_2$ & $-1/3$ & $1/3$ \\
    \bottomrule
\end{tabular}
}
$$A = \begin{bmatrix*}[r]
    2/3 & -1/3 \\
    -1/3 & 2/3
\end{bmatrix*}$$
A la matriz anterior se le llama \textbf{matriz de transición} de la base $\mathcal{B}_1$ a la base $\mathcal{B}_2$. De esta manera,
\begin{align*}
    \mathbb{v}_{\mathcal{B}_2} & = A \mathbb{v}_{\mathcal{B}_1} \\
    & = \begin{bmatrix*}[r]
        -2/3 & -1/3 \\
        -1/3 & 2/3
    \end{bmatrix*} \begin{pmatrix}
        4 \\
        3
    \end{pmatrix} \\
    & = \begin{pmatrix}
        5/3 \\
        2/3
    \end{pmatrix}
\end{align*}
Además, observemos que
\begin{equation*}
    A^{-1} = \left[\begin{array}{cc}
        \tikzmarkin[ver=style azull]{col 1-b}2 & \tikzmarkin[ver=style azull]{col 2-b}1 \\
        1 \tikzmarkend{col 1-b} & 2\tikzmarkend{col 2-b} \\
    \end{array}\right]
    \tikz[overlay, remember picture]{
    \node[below=20pt of col 1-b.south west](A) {};
    \node[right=2pt of A] (B) {};
    \node[below=20pt of B] (C) {$\mathbb{u}_1$};
    \draw[-latex] (C) -- (B);

    \node[below=20pt of col 2-b.south west](D) {};
    \node[right=2pt of D] (E) {};
    \node[below=20pt of E] (F) {$\mathbb{u}_2$};
    \draw[-latex] (F) -- (E);
    }
\end{equation*}
\,\\ \,\\

Entonces, para cambiar de la base $\mathcal{B}_1$ a $\mathcal{B}_2$ usaremos la matriz $A$ y para cambiar de la base $\mathcal{B}_2$ a $\mathcal{B}_1$ usaremos la matriz $A^{-1}$.

\begin{example}
    Determine la matriz de transición de $\mathcal{B}_1 = \left\{ \begin{pmatrix}
        1 \\
        1
    \end{pmatrix},  \begin{pmatrix}
        2 \\
        3
    \end{pmatrix} \right\}$ a $\mathcal{B}_2 = \left\{ \begin{pmatrix}
        0 \\
        3
    \end{pmatrix},  \begin{pmatrix*}[r]
        5 \\
        -1
    \end{pmatrix*} \right\}$. \\
    \solucion Se quiere determinar la matriz $A$ de transición tal que
    \begin{center}
        \begin{tikzpicture}
            \node at (0,0) {$\mathcal{B}_1$};
            \node at (4,0) {$\mathcal{B}_2$};
            \draw[->,>=latex] (0.3,0) -- (3.7,0);
            \node[above] at (2,0) {$A$};
        \end{tikzpicture}
    \end{center}
    Sea $\mathcal{E} = \left\{ e_1,  e_2 \right\}$ la base canónica de $\RR[2]$, observemos que
    \begin{center}
        \begin{tikzpicture}[scale=1.2]
            \node at (0,0) {$\mathcal{E}$};
            \node at (4,0) {$\mathcal{E}$};
            \node at (0,2) {$\mathcal{B}_1$};
            \node at (4,2) {$\mathcal{B}_2$};
            \node[above] at (2,0) {$I$};
            \node[above] at (2,2) {$A$};
            \node[left] at (0,1) {$J$};
            \node[left] (A) at (3.9,1) {$H$};
            \node[right of = A,yshift=0.6] (B) {$H^{-1}$};
            \draw[->,>=latex] (0.3,0) -- (3.7,0);
            \draw[->,>=latex] (0.3,2) -- (3.7,2);
            \draw[->,>=latex] (0,1.7) -- (0,0.3);
            \draw[->,>=latex] (3.9,1.7) -- (3.9,0.3);
            \draw[<-,>=latex] (4.1,1.7) -- (4.1,0.3);
        \end{tikzpicture}
        %\captionof{figure}{~}
    \end{center}\newpage\noindent
    siendo $J = \begin{bmatrix}
        1 & 2 \\
        1 & 3
    \end{bmatrix}$ y $H = \begin{bmatrix*}[r]
        0 & -5 \\
        3 & -1
    \end{bmatrix*}$. Del diagrama anterior,
    $$A = H^{-1}IJ$$
    Notemos que ya tenemos $I$ y $J$, solo nos falta encontrar $H^{-1}$, pero es fácil determinar que
    $$H^{-1} = \begin{bmatrix*}[r]
        1/15 & 1/3 \\
        1/5 & 0
    \end{bmatrix*}$$
    Por tanto,
    \begin{align*}
        A & = H^{-1}IJ \\
        & = \begin{bmatrix*}[r]
            1/15 & 1/3 \\
            1/5 & 0
        \end{bmatrix*} \begin{bmatrix}
            1 & 0 \\
            0 & 1
        \end{bmatrix} \begin{bmatrix}
            1 & 2 \\
            1 & 3
        \end{bmatrix} \\
        & = \begin{bmatrix*}[r]
            2/5 & 17/15 \\
            1/5 & 2/5
        \end{bmatrix*}
    \end{align*}
    Por ejemplo, imaginemos que queremos expresar el vector $\begin{pmatrix*}[r]
        2 \\
        -1
    \end{pmatrix*}$ de la base $\mathcal{B}_1$ en la base $\mathcal{B}_2$, entonces
    $$\begin{bmatrix*}[r]
        2/5 & 17/15 \\
        1/5 & 2/5
    \end{bmatrix*} \begin{pmatrix*}[r]
        2 \\
        -1
    \end{pmatrix*}_{\mathcal{B}_1} = \begin{pmatrix*}[r]
        -1/3 \\
        0
    \end{pmatrix*}_{\mathcal{B}_2}$$
\end{example}

\section*{Generalizaciones}

Ahora, considere $\mathcal{B}_1$ y $\mathcal{B}_2$ cualesquiera dos bases (distintas a la canónica) en $\RR[n]$. Es decir, $\mathcal{B}_1 = \left\{ \mathbb{v}_1, \mathbb{v}_2, \dots, \mathbb{v}_n \right\}$ y $\mathcal{B}_2 = \left\{ \mathbb{w}_1, \mathbb{w}_2, \dots, \mathbb{w}_n \right\}$, y se quiere determinar la matriz de transición de la base $\mathcal{B}_1$ a la base $\mathcal{B}_2$. Consideremos el diagrama
\begin{center}
    \begin{tikzpicture}[scale=1.2]
        \node at (0,0) {$\mathcal{E}$};
        \node at (4,0) {$\mathcal{E}$};
        \node at (0,2) {$\mathcal{B}_1$};
        \node at (4,2) {$\mathcal{B}_2$};
        \node[above] at (2,0) {$I$};
        \node[above] at (2,2) {$A$};
        \node[left] at (0,1) {$J$};
        \node[left] (A) at (3.9,1) {$H$};
        \node[right of = A,yshift=0.6] (B) {$H^{-1}$};
        \draw[->,>=latex] (0.3,0) -- (3.7,0);
        \draw[->,>=latex] (0.3,2) -- (3.7,2);
        \draw[->,>=latex] (0,1.7) -- (0,0.3);
        \draw[->,>=latex] (3.9,1.7) -- (3.9,0.3);
        \draw[<-,>=latex] (4.1,1.7) -- (4.1,0.3);
    \end{tikzpicture}
    %\captionof{figure}{~}
\end{center}
siendo $\mathcal{E} = \left\{ e_1, e_2, \dots, e_n \right\}$. De esta forma, la matriz de transición $A$ está dada por
$$A = H^{-1}IJ$$

\begin{definition}
    La matriz de transición $A$ de la base $\mathcal{B}_1$ a $\mathcal{B}_2$ se define como la matriz cuyas columnas consisten en las coordenadas de los vectores de la base $\mathcal{B}_2$ expresados en términos de la base $\mathcal{B}_1$. 
\end{definition}

\begin{theorem}
    Sea $\mathcal{B}_1$ y $\mathcal{B}_2$ bases de un espacio vectorial $V$. Sea $A$ la matriz de transición de $\mathcal{B}_1$ a $\mathcal{B}_2$. Entonces para toda $\mathbb{x} \in V$,
    $$\mathbb{x}_{\mathcal{B}_2} = A\mathbb{x}_{\mathcal{B}_1}$$
\end{theorem}

\begin{theorem}
    Sea $A$ la matriz de transición de $\mathcal{B}_1$ a $\mathcal{B}_2$. Entonces $A^{-1}$ es la matriz de transición de $\mathcal{B}_2$ a $\mathcal{B}_1$.
\end{theorem}

\newpage
Sean $V$ y $W$ dos espacios vectoriales sobre $K$, con $\mathcal{B}_1 = \left\{ \mathbb{v}_1, \mathbb{v}_2, \dots, \mathbb{v}_n \right\}$ base de $V$ y $\mathcal{B}_2 = \left\{ \mathbb{w}_1, \mathbb{w}_2, \dots, \mathbb{w}_m \right\}$ base de $W$. Sea $T: V \longrightarrow W$ una transformación lineal tal que
$$T\mathbb{v}_1 = \chi_1, T\mathbb{v}_2 = \chi_2, \dots, T\mathbb{v}_n = \chi_n$$
Sea además
$$\chi_j = a_{1j}\mathbb{w}_1 + a_{2j}\mathbb{w}_2 + \cdots + a_{mj}\mathbb{v}_m, \quad \text{para } j = 1, 2, \dots, n$$
Ahora
$$(\chi_1)_{\mathcal{B}_2} = \begin{bmatrix} a_{11} \\ a_{21} \\ \vdots \\ a_{m1} \end{bmatrix}, (\chi_2)_{\mathcal{B}_2} = \begin{bmatrix} a_{12} \\ a_{22} \\ \vdots \\ a_{m2} \end{bmatrix}, \dots, (\chi_n)_{\mathcal{B}_2} = \begin{bmatrix} a_{1n} \\ a_{2n} \\ \vdots \\ a_{mn} \end{bmatrix}$$
Sea
\begin{align*}
    A & = \begin{bmatrix}
        (\chi_1)_{\mathcal{B}_2} & (\chi_2)_{\mathcal{B}_2} & \cdots & (\chi_n)_{\mathcal{B}_2}
    \end{bmatrix} \\
    & = \begin{bmatrix}
        a_{11} & a_{12} & \cdots & a_{1n} \\
        a_{21} & a_{22} & \cdots & a_{2n} \\
        \vdots & & \ddots & \\
        a_{m1} & a_{m2} & \cdots & a_{mn}
    \end{bmatrix}
\end{align*}
Además,
$$(\mathbb{v}_1)_{\mathcal{B}_1} = \begin{bmatrix}
    1 \\
    0 \\
    \vdots \\
    0 \\
    0
\end{bmatrix}$$
En general
\begin{center}
    \begin{tikzpicture}

        \node (A) at (0,0) {
        $(\mathbb{v}_j)_{\mathcal{B}_1} = \begin{bmatrix}
            0 \\
            0 \\
            \vdots \\
            1 \\
            \vdots \\
            0 \\
            0
        \end{bmatrix}$
        };
        \node[right = 30pt of A.east](L)  {$j$-ésimo elemento};
        \draw[->, >=stealth, shorten > =12pt, thick] (L.west) -- ($(A.east) + (-0.5,0)$);
    \end{tikzpicture}
\end{center}
entonces
\begin{align*}
    A(\mathbb{v}_j)_{\mathcal{B}_1} & = \begin{bmatrix}
        a_{1j} \\
        a_{2j} \\
        \vdots \\
        a_{mj}
    \end{bmatrix} \\
    & = (\chi_j)_{\mathcal{B}_2} \\
    & = (T\mathbb{v}_j)_{\mathcal{B}_2}
\end{align*}
Por tanto,
$$(T\mathbb{v}_j)_{\mathcal{B}_2} = A(\mathbb{v}_j)_{\mathcal{B}_1}, \text{ para } j = 1, 2, \dots, n$$
Ahora, sea $\mathbb{v} \in V$, entonces
$$\mathbb{v} = \alpha_1\mathbb{v}_1 + \alpha_2\mathbb{v}_2 + \cdots + \alpha_n\mathbb{v}_n$$
siendo $\alpha_i \in K$. Demostremos que
$$A(\mathbb{v})_{\mathcal{B}_1} = (T\mathbb{v})_{\mathcal{B}_2}$$
\newpage\noindent
Entonces
\begin{align*}
    A(\mathbb{v})_{\mathcal{B}_1} & = A(\alpha_1\mathbb{v}_1 + \alpha_2\mathbb{v}_2 + \cdots + \alpha_n\mathbb{v}_n)_{\mathcal{B}_1} \\
    & = A \big( \alpha_1(\mathbb{v}_1)_{\mathcal{B}_1} + \alpha_2(\mathbb{v}_2)_{\mathcal{B}_1} + \cdots + \alpha_n(\mathbb{v}_n)_{\mathcal{B}_1} \big) \\
    & = A \big( \alpha_1(\mathbb{v}_1)_{\mathcal{B}_1} \big) + A \big( \alpha_2(\mathbb{v}_2)_{\mathcal{B}_1} \big) + \cdots + A \big( \alpha_n(\mathbb{v}_n)_{\mathcal{B}_1} \big) \\
    & = \alpha_1 \big( A(\mathbb{v}_1)_{\mathcal{B}_1} \big) + \alpha_2 \big( A(\mathbb{v}_2)_{\mathcal{B}_1} \big) + \cdots + \alpha_n \big( A(\mathbb{v}_n)_{\mathcal{B}_1} \big) \\
    & = \alpha_1 (T\mathbb{v}_1)_{\mathcal{B}_2} + \alpha_2 (T\mathbb{v}_2)_{\mathcal{B}_2} + \cdots + \alpha_n (T\mathbb{v}_n)_{\mathcal{B}_2} \\
    & = \big( T(\alpha_1\mathbb{v}_1) \big)_{\mathcal{B}_2} + \big( T(\alpha_2\mathbb{v}_2) \big)_{\mathcal{B}_2} + \cdots + \big( T(\alpha_n\mathbb{v}_n) \big)_{\mathcal{B}_2} \\
    & = \big( T(\alpha_1\mathbb{v}_1) + T(\alpha_2\mathbb{v}_1) + \cdots + T(\alpha_n\mathbb{v}_n) \big)_{\mathcal{B}_2} \\
    & = \big( T(\alpha_1\mathbb{v}_1 + \alpha_2\mathbb{v}_2 + \cdots + \alpha_n\mathbb{v}_n) \big)_{\mathcal{B}_2} \\
    & = (T\mathbb{v})_{\mathcal{B}_2}
\end{align*}

\begin{example}
    Sea $T:\RR[2] \longrightarrow \RR[2]$ definida por $T\begin{pmatrix} x \\ y \end{pmatrix} = \begin{pmatrix} x-2y \\ 2x+y \end{pmatrix}$, siendo $\mathcal{B}_1 = \left\{ \begin{pmatrix*}[r] 1 \\ -2 \end{pmatrix*}, \begin{pmatrix} 3 \\ 2 \end{pmatrix} \right\} = \mathcal{B}_2$. Determinemos la matriz $A$ tal que
    $$A(\mathbb{v})_{\mathcal{B}_1} = (T\mathbb{v})_{\mathcal{B}_2}, \forall \mathbb{v} \in V$$
    Sean $\mathbb{v}_1 = \begin{pmatrix*}[r] 1 \\ -2 \end{pmatrix*}$, $\mathbb{v}_2 = \begin{pmatrix} 3 \\ 2 \end{pmatrix}$ elementos de la base $\mathcal{B}_1$ y sean $\mathbb{w}_1 = \begin{pmatrix*}[r] 1 \\ -2 \end{pmatrix*}$, $\mathbb{w}_2 = \begin{pmatrix} 3 \\ 2 \end{pmatrix}$ elementos de la base $\mathcal{B}_2$. Ahora
    \begin{align*}
        (T\mathbb{v}_1)_{\mathcal{B}_2} = \begin{pmatrix} 5 \\ 0 \end{pmatrix} & = a\mathbb{w}_1 + b\mathbb{w}_2 \\
        & = a \begin{pmatrix*}[r] 1 \\ -2 \end{pmatrix*} + b\begin{pmatrix} 3 \\ 2 \end{pmatrix} \\
        & = \begin{pmatrix} a+3b \\ -2a+2b \end{pmatrix}
    \end{align*}
    entonces obtenemos el siguiente sistema
    \begin{align*}
        a + 3b & = 5 \\
        -2a + 2b & = 0
    \end{align*}
    De la segunda ecuación se obtiene que $a = b$ y sustituyendo en la primer ecuación se obtiene que $b = 5/4$, de donde se sigue que $a = 5/4$. Entonces
    $$T(\mathbb{v}_1)_{\mathcal{B}_2} = \begin{bmatrix} 5/4 \\ 5/4 \end{bmatrix}$$
    De manera análoga,
    \begin{align*}
        (T\mathbb{v}_2)_{\mathcal{B}_2} = \begin{pmatrix*}[r] -1 \\ 8 \end{pmatrix*} & = a\mathbb{w}_1 + b\mathbb{w}_2 \\
        & = a \begin{pmatrix*}[r] 1 \\ -2 \end{pmatrix*} + b\begin{pmatrix} 3 \\ 2 \end{pmatrix} \\
        & = \begin{pmatrix} a+3b \\ -2a+2b \end{pmatrix}
    \end{align*}
    de donde se obtiene que $b = 3/4$ y $a = -13/4$. Entonces
    $$T(\mathbb{v}_2)_{\mathcal{B}_2} = \begin{bmatrix*}[r] -13/4 \\ 3/4 \end{bmatrix*}$$
    Por lo tanto,
    $$A = \begin{bmatrix*}[r]
        5/4 & -13/4 \\
        5/4 & 3/4
    \end{bmatrix*}$$
\end{example}

\newpage

\section{Ejercicios}

\noindent
De los problemas 1 al 39 determine si la transformación de $V$ en $W$ dada es lineal.

\begin{tasks}[
    style=enumerate,
    label-offset = 3mm,
    ](2)
    \task $T: \RR[2] \longrightarrow \RR$; $T\begin{pmatrix*}x \\ y\end{pmatrix*}=x$
    \task $T: \RR[2] \longrightarrow \RR[2]$; $T\begin{pmatrix*}x \\ y\end{pmatrix*}=\begin{pmatrix*}x \\ 0\end{pmatrix*}$
    \task $T: \RR[2] \longrightarrow \RR[2]$; $T\begin{pmatrix*}x \\ y\end{pmatrix*}=\begin{pmatrix*}1 \\ y\end{pmatrix*}$
    \task $T: \RR[2] \longrightarrow \RR$; $T\begin{pmatrix*}x \\ y\end{pmatrix*}=x+1$
    \task $T: \RR[3] \longrightarrow \RR[2]$; $T\begin{pmatrix*}x \\ y \\ z\end{pmatrix*}=\begin{pmatrix*}x \\ y\end{pmatrix*}$
    \task $T: \RR[3] \longrightarrow \RR[2]$; $T\begin{pmatrix*}x \\ y \\ z\end{pmatrix*}=\begin{pmatrix*}0 \\ y\end{pmatrix*}$
    \task $T: \RR[3] \longrightarrow \RR[2]$; $T\begin{pmatrix*}x \\ y \\ z\end{pmatrix*}=\begin{pmatrix*}1 \\ z\end{pmatrix*}$
    \task $T: \RR[3] \longrightarrow \RR[2]$; $T\begin{pmatrix*}x \\ y \\ z\end{pmatrix*}=\begin{pmatrix*}x \\ y+z\end{pmatrix*}$
    \task $T: \RR[2] \longrightarrow \RR[2]$; $T\begin{pmatrix*}x \\ y\end{pmatrix*}=\begin{pmatrix*}x^{2} \\ y^{2}\end{pmatrix*}$
    \task $T: \RR[2] \longrightarrow \RR[2]$; $T\begin{pmatrix*}x \\ y\end{pmatrix*}=\begin{pmatrix*}x \\ x/y\end{pmatrix*}$
    \task $T: \RR[2] \longrightarrow \RR[2]$; $T\begin{pmatrix*}x \\ y\end{pmatrix*}=\begin{pmatrix*}y \\ x\end{pmatrix*}$
    \task $T: \RR[2] \longrightarrow \RR[4]$; $T\begin{pmatrix*}x \\ y\end{pmatrix*}=\begin{pmatrix*}x \\ x+y \\ y \\ x-y\end{pmatrix*}$
    \task $T: \RR[2] \longrightarrow \RR$; $T\begin{pmatrix*}x \\ y\end{pmatrix*}=x y$
    \task \!$T: \RR[n] \longrightarrow \RR[2]$; $T\begin{pmatrix*}x_{1} \\ x_{2} \\ \vdots \\ x_{n}\end{pmatrix*}=\begin{pmatrix*} |x_{4}| \\ x_{1} \end{pmatrix*}$
    \task*(2) $T: \RR[n] \longrightarrow \RR$; $T\begin{pmatrix*}x_{1} \\ x_{2} \\ \vdots \\ x_{n}\end{pmatrix*}=x_{1}+x_{2}+\cdots+x_{n}$
    \task $T: \RR \longrightarrow \RR[n]$; $T(x)=\begin{pmatrix*}x \\ x \\ \vdots \\ x\end{pmatrix*}$
    \task $T: \RR[4] \longrightarrow \RR[2]$; $T\begin{pmatrix*}x \\ y \\ z \\ w\end{pmatrix*}=\begin{pmatrix*}x+z \\ y+w\end{pmatrix*}$
\end{tasks}
\begin{enumerate}[start=18]
    \item $T: \RR[4] \longrightarrow \mathcal{M}_{2 \times 2}(\RR)$; $T\begin{pmatrix*}x \\ y \\ z \\ w\end{pmatrix*}=\begin{pmatrix*}x & z \\ y & w\end{pmatrix*}$
    \item $T: \mathcal{M}_{n \times n}(\RR) \longrightarrow \mathcal{M}_{n \times n}(\RR)$; $T(A)=A B$, donde $B$ es una matriz fija de $n \times n$
    \item $T: \mathcal{M}_{n \times n}(\RR) \longrightarrow \mathcal{M}_{n \times n}(\RR)$; $T(A)=A^{T} A$
    \item $T: \mathcal{M}_{p \times q}(\RR) \longrightarrow \mathcal{M}_{p \times q}(\RR)$; $T(A)=A^{T}$
    \item $T: \mathcal{M}_{m \times n}(\RR) \longrightarrow \mathcal{M}_{q \times n}(\RR)$; $T(A)=B A$, donde $B$ es una matriz fija de $q \times m$
    \item $T: D_{n} \longrightarrow D_{n}$; $T(D)=D^{2}$\infoBulle{$D_{n}$ es el conjunto de matrices diagonales de $n \times n$}
    \item $T: D_{5} \longrightarrow \RR[3]$; $T(D)=\begin{pmatrix*}d_{11}+2 d_{33} \\ d_{22}-3 d_{33} \\ d_{55}\end{pmatrix*}$
    \item $T: P_{2} \longrightarrow P_{1}$; $T\left(a_{0}+a_{1} x+a_{2} x^{2}\right)=a_{0}+a_{1} x$
    \item $T: P_{2} \longrightarrow P_{1}$; $T\left(a_{0}+a_{1} x+a_{2} x^{2}\right)=a_{1}+a_{2} x$
    \item $T: P_{3} \longrightarrow \mathcal{M}_{2 \times 2}(\RR)$; $T\left(a_{0}+a_{1} x+a_{2} x^{2}+a_{3} x^{3}\right)=\begin{bmatrix*}a_{0}+a_{1} & a_{1}+a_{2} \\ a_{2}+a_{3} & a_{3}+a_{0}\end{bmatrix*}$
    \item $T: \RR \longrightarrow P_{n}$; $T(a)=a+a x+a x^{2}+\cdots+a x^{n}$
    \item $T: P_{2} \longrightarrow P_{4}$; $T\big(p(x)\big)=[p(x)]^{2}$
    \item $T: P_{2} \longrightarrow P_{4}$; $T\big(p(x)\big)=p(x)+x^{2} p(x)$
    \item $T: C[0,1] \longrightarrow C[0,1]$; $T_f=f^{2}(x)$
    \item $T: C[0,1] \longrightarrow C[0,1]$; $T_f=f(x)+1$
    \item $T: C[0,1] \longrightarrow C[0,1]$; $T_f=x^{2} f(x)+x f(x)$
    \item $T: C[0,1] \longrightarrow \RR$; $\displaystyle T_f=\int_{0}^{1} f(x) g(x) d x$, donde $g$ es una función fija en $C[0,1]$
    \item $T: C^{1}[0,1] \longrightarrow C[0,1]$; $\displaystyle T_f=\frac{d}{d x}\big(f(x) g(x)\big)$, donde $g(x)$ es una función fija en $C^{1}[0,1]$
    \item $T: C[0,1] \longrightarrow C[1,2]$; $T_f=f(x-1)$
    \item $T: C[0,1] \longrightarrow \RR$; $\displaystyle T_f=f\left(\frac{1}{2}\right)$
    \item $T: C^{1}[0,1] \longrightarrow \RR$; $\displaystyle T_f=\left.\left(\frac{d}{d x} f(x)\right)\right|_{x=1/2}$
    \item $T: \mathcal{M}_{n \times n}(\RR) \longrightarrow \RR$; $T(A)=\Det A$
    \item Sea $T: \RR[2] \longrightarrow \RR[2]$ dado por $T\begin{pmatrix} x \\ y \end{pmatrix} = \begin{pmatrix*}[r] -x \\ -y \end{pmatrix*}$. Describa $T$ geométricamente.
    \item Sea $T$ una transformación lineal de $\RR[2] \longrightarrow \RR[3]$ tal que $T\begin{pmatrix*}1 \\ 0\end{pmatrix*}=\begin{pmatrix*}1 \\ 2 \\ 3\end{pmatrix*}$ y $T\begin{pmatrix*}0 \\ 1\end{pmatrix*}=\begin{pmatrix*}[r]-4 \\ 0 \\ 5\end{pmatrix*}$. Encuentre:
    \begin{tasks}(2)
        \task $T\begin{pmatrix*}2 \\ 4\end{pmatrix*}$
        \task $T\begin{pmatrix*}[r]-3 \\ 7\end{pmatrix*}$
    \end{tasks}
    \item Suponga que en un espacio vectorial real $V, T$ satisface $T(\mathbb{x}+\mathbb{y})=T \mathbb{x}-T \mathbb{y}$ y $T(\alpha \mathbb{x})=$ $\alpha T \mathbb{x}$ para $\alpha \geq 0$. Demuestre que $T$ es lineal.
    \item Encuentre una transformación lineal $T: \mathcal{M}_{3 \times 3}(\RR) \longrightarrow \mathcal{M}_{2 \times 2}(\RR)$.
    \item Si $T$ es una transformación lineal de $V$ en $W$, demuestre que $T(\mathbb{x}-\mathbb{y})=T \mathbb{x}-T \mathbb{y}$.
    \item Si $T$ es una transformación lineal de $V$ en $W$, demuestre que $T \mathbb{0}=\mathbb{0}$. ¿Son estos dos vectores cero el mismo?
    \item Sean $V$ y $W$ dos espacios vectoriales. Denote por $\mathcal{L}(V, W)$ el conjunto de transformaciones lineales de $V$ en $W$. Si $T_{1}$ y $T_{2}$ están en $\mathcal{L}(V, W)$, defina $\alpha T_{1}$ y $T_{1}+T_{2}$ por $\left(\alpha T_{1}\right) \mathbb{v}=$ $\alpha\left(T_{1} \mathbb{v}\right)$ y $\left(T_{1}+T_{2}\right) \mathbb{v}=T_{1} \mathbb{v}+T_{2} \mathbb{v}$. Pruebe que $\mathcal{L}(V, W)$ es un espacio vectorial.
\end{enumerate}\newpage\noindent
De los problemas 47 al 59 encuentre núcleo, imagen, rango y nulidad de la transformación lineal dada.
\begin{tasks}[
    start=47,
    style=enumerate,
    label-offset = 3mm,
    ](2)
    \task $T: \RR[2] \longrightarrow \RR$; $T\begin{pmatrix}x \\ y\end{pmatrix}=x$
    \task $T: \RR[2] \longrightarrow \RR[2]$; $T\begin{pmatrix}x \\ y\end{pmatrix}=\begin{pmatrix}x \\ 0\end{pmatrix}$
    \task $T: \RR[3] \longrightarrow \RR[2]$; $T\begin{pmatrix}x \\ y \\ z\end{pmatrix}=\begin{pmatrix}z \\ y\end{pmatrix}$
    \task $T: \RR[2] \longrightarrow \RR[2]$; $T\begin{pmatrix}x \\ y\end{pmatrix}=\begin{pmatrix*}[r]-4 y \\ y\end{pmatrix*}$
    \task $T: \RR[2] \longrightarrow \RR$; $T\begin{pmatrix}x \\ y\end{pmatrix}=x+y$
    \task $T: \RR[4] \longrightarrow \RR[2]$; $T\begin{pmatrix}x \\ y \\ z \\ w\end{pmatrix}=\begin{pmatrix}x+z \\ y+w\end{pmatrix}$
\end{tasks}
\begin{enumerate}[start=53]
    \item $T: \mathcal{M}_{2 \times 2}(\RR) \longrightarrow \mathcal{M}_{2 \times 2}(\RR)$; $T(A)=B A$, donde $B=\begin{bmatrix}1 & 0 \\ 3 & 1\end{bmatrix}$
    \item $T: \RR \longrightarrow P_3$; $T(a)=a+a x+a x^2+a x^3$.
    \item $T: \RR[2] \longrightarrow P_3$; $T\begin{pmatrix}a \\ b\end{pmatrix}=a+b x+(a+b) x^2+(a-b) x^3$
    \item $T: \mathcal{M}_{n \times n}(\RR) \longrightarrow \mathcal{M}_{n \times n}(\RR)$; $T(A)=A^{T}+A$
    \item $T: C^1[0,1] \longrightarrow C[0,1]$; $T_f=f^{\prime}$
    \item $T: C^2[0,1] \longrightarrow C[0,1]$; $T_f=f^{\prime \prime}$
    \item $T: C[0,1] \longrightarrow \RR$; $T_f=f(0)$
    \item Sea $T: V \longrightarrow W$ una transformación lineal, sea $\left\{\mathbb{v}_1, \mathbb{v}_2, \dots, \mathbb{v}_n\right\}$ una base para $V$ y suponga que $T \mathbb{v}_i=\mathbb{0}$ para $i=1,2, \dots, n$. Demuestre que $T$ es la transformación cero.
    \item Encuentre todas las transformaciones lineales de $\RR[2]$ en $\RR[2]$ tales que la recta $y=0$ se transforma en la recta $x=0$.
    \item Encuentre todas las transformaciones lineales de $\RR[2]$ en $\RR[2]$ que llevan a la recta $y=a x$ a la recta $y=b x$.
    \item Encuentre una transformación lineal $T$ de $\RR[3] \longrightarrow \RR[3]$ tal que
    $$\Nuc T=\left\{ \begin{pmatrix} x \\ y \\ z \end{pmatrix} \mid 2 x-y+z=0\right\} .$$
    \item Encuentre una transformación lineal $T$ de $\RR[3] \longrightarrow \RR[3]$ tal que
    $$\Ima T=\left\{\begin{pmatrix} x \\ y \\ z \end{pmatrix} \mid 3 x+2 y-5 z=0\right\} .$$
\end{enumerate}
De los problemas 65 al 86 encuentre la representación matricial $A_{T}$ de la transformación lineal $T$, $\Nuc T$, $\Ima T$, $\nu(T)$ y $\rho(T)$. A menos que se especifique otra cosa, suponga que $\mathcal{B}_1$ y $\mathcal{B}_2$ son bases canónicas.
\begin{enumerate}[resume]
    \item $T: \RR[2] \longrightarrow \RR$; $T\begin{pmatrix*}x \\ y\end{pmatrix*}=3 x-2 y$
    \item $T: \RR[2] \longrightarrow \RR[2]$; $T\begin{pmatrix*}x \\ y\end{pmatrix*}=\begin{pmatrix*}x-2 y \\ -x+y\end{pmatrix*}$
    \item $T: \RR[2] \longrightarrow \RR[3]$; $T\begin{pmatrix*}x \\ y\end{pmatrix*}=\begin{pmatrix*}x+y \\ x-y \\ 2 x+3 y\end{pmatrix*}$\newpage
    \item $T: \RR[2] \longrightarrow \RR[2]$; $T\begin{pmatrix*}x \\ y\end{pmatrix*}=\begin{pmatrix*}y \\ x\end{pmatrix*}$
    \item $T: \RR[3] \longrightarrow \RR[2]$; $T\begin{pmatrix*}x \\ y \\ z\end{pmatrix*}=\begin{pmatrix*}x-y+z \\ -2 x+2 y-2 z\end{pmatrix*}$
    \item $T: \RR[2] \longrightarrow \RR[2]$; $T\begin{pmatrix*}x \\ y\end{pmatrix*}=\begin{pmatrix*}a x+b y \\ c x+d y\end{pmatrix*}$
    \item $T: \RR[2] \longrightarrow \RR[3]$; $T\begin{pmatrix*}x \\ y\end{pmatrix*}=\begin{pmatrix*}x+y \\ 3 x-2 y \\ y-x\end{pmatrix*}$
    \item $T: \RR[3] \longrightarrow \RR[3]$; $T\begin{pmatrix*}x \\ y \\ z\end{pmatrix*}=\begin{pmatrix*}[r]x-y+2 z \\ 3 x+y+4 z \\ 5 x-y+8 z\end{pmatrix*}$
    \item $T: \RR[3] \longrightarrow \RR[3]$; $T\begin{pmatrix*}x \\ y \\ z\end{pmatrix*}=\begin{pmatrix*}-x+2 y+z \\ 2 x-4 y-2 z \\ -3 x+6 y+3 z\end{pmatrix*}$
    \item $T: \RR[4] \longrightarrow \RR[2]$; $T\begin{pmatrix*}x \\ y \\ z \\ w\end{pmatrix*}=\begin{pmatrix*}x+z \\ 5 w-4 y\end{pmatrix*}$
    \item $T: \RR[4] \longrightarrow \RR[4]$; $T\begin{pmatrix*}x \\ y \\ z \\ w\end{pmatrix*}=\begin{pmatrix*}x-y+2 z+w \\ -x+z+2 w \\ x-2 y+5 z+4 w \\ 2 x-y+z-w\end{pmatrix*}$
    \item $T: \RR[4] \longrightarrow \RR[2]$; $T\begin{pmatrix*}w \\ x \\ y \\ z\end{pmatrix*}=\begin{pmatrix*}a w+b x \\ c y+d z\end{pmatrix*}$
    \item $T: \RR[2] \longrightarrow \RR[2]$; $T\begin{pmatrix*}x \\ y\end{pmatrix*}=\begin{pmatrix*}[r]3 x+2 y \\ -5 x-4 y\end{pmatrix*}$; $\mathcal{B}_1=\mathcal{B}_2=\left\{\begin{pmatrix*}[r]3 \\ -2\end{pmatrix*},\begin{pmatrix*}[r]-1 \\ 1\end{pmatrix*}\right\}$
    \item $T: \RR[2] \longrightarrow \RR[2]$; $T\begin{pmatrix*}x \\ y\end{pmatrix*}=\begin{pmatrix*}4 x-y \\ 3 x+2 y\end{pmatrix*}$; $\mathcal{B}_1=\mathcal{B}_2=\left\{\begin{pmatrix*}[r]-1 \\ 1\end{pmatrix*},\begin{pmatrix*}4 \\ 3\end{pmatrix*}\right\}$
    \item $T: P_{2} \longrightarrow P_{3}$; $T\left(a_{0}+a_{1} x+a_{2} x^{2}\right)=a_{1}-a_{1} x+a_{0} x^{3}$
    \item $T: \RR \longrightarrow P_{3}$; $T(a)=a+a x+a x^{2}+a x^{3}$
    \item $T: P_{2} \longrightarrow \RR[2]$; $T\left(a_{0}+a_{1} x+a_{2} x^{2}\right)=\begin{pmatrix*}a_{0}+a_{1} \\ a_{1}+a_{2}+a_{3}\end{pmatrix*}$
    \item $T: P_{3} \longrightarrow P_{1}$; $T\left(a_{0}+a_{1} x+a_{2} x^{2}+a_{3} x^{3}\right)=\left(a_{1}+a_{3}\right) x-a_{2}$
    \item $T: P_{4} \longrightarrow P_{4}$; $T\left(a_{0}+a_{1} x+a_{2} x^{2}+a_{3} x^{3}+a_{4} x^{4}\right)=a_{4} x^{4}+a_{2} x^{2}+a_{0}$
    \item $T: \mathcal{M}_{2 \times 2}(\RR) \longrightarrow \mathcal{M}_{2 \times 2}(\RR)$; $T\begin{bmatrix*}a & b \\ c & d\end{bmatrix*}=\begin{bmatrix*}a-b+2 c+d & -a+2 c+2 d \\ a-2 b+5 c+4 d & 2 a-b+c-d\end{bmatrix*}$
    \item $T: P_{4} \longrightarrow P_{3}$; $T\left(a_{0}+a_{1} x+a_{2} x^{2}+a_{3} x^{3}+a_{4} x^{4}\right)=a_{3} x^{3}+a_{1} x$
    \item $T: \mathcal{M}_{2 \times 3}(\RR) \longrightarrow \RR[3]$; $T\begin{bmatrix*}a & b & c \\ d & e & f\end{bmatrix*}=\begin{pmatrix*}a+e \\ b+f \\ c+d\end{pmatrix*}$
\end{enumerate}