\chapter*{PREFACIO}
\addcontentsline{toc}{chapter}{PREFACIO}

En el ámbito de las ciencias matemáticas, el álgebra lineal emerge como una disciplina fundamental cuya influencia se extiende a diversas áreas del conocimiento. Desde su concepción en los albores del desarrollo matemático hasta su aplicación en campos tan variados como la física teórica, la ingeniería, la economía y la inteligencia artificial, el álgebra lineal ha probado ser un pilar inquebrantable en la comprensión y modelado de fenómenos complejos.

Al abordar la redacción de este texto, he tenido en cuenta la necesidad de equilibrar la claridad expositiva con la profundidad conceptual. Cada capítulo se estructura de manera que los fundamentos se presenten de forma gradual y sistemática, permitiendo al lector asimilar los conceptos básicos antes de adentrarse en temas más complejos. He incorporado numerosos ejemplos, ilustraciones y problemas resueltos para enriquecer la comprensión y facilitar la asimilación de los contenidos.

Este libro también pretende fomentar una apreciación más profunda del álgebra lineal como una disciplina en constante evolución, cuyas aplicaciones van más allá del ámbito académico y se extienden a la resolución de problemas prácticos en la vida cotidiana y en la investigación científica de vanguardia. Se alienta a los lectores a explorar conexiones interdisciplinarias y a aplicar los conceptos aprendidos en contextos diversos, cultivando así una comprensión holística de la materia. El propósito de este libro es servir como recurso educativo y de referencia, cada página de este volumen ha sido meticulosamente diseñada para guiar al lector a través de un viaje intelectual que abarca desde los conceptos más elementales hasta las estructuras más abstractas y sofisticadas.

Esta obra, aunque extensa, está dirigida a estudiantes de Ingeniería, que buscan ampliar su comprensión de esta área de las matemáticas y obtener herramientas prácticas para resolver problemas complejos de la vida real. En este libro se presentan los principios fundamentales de la matemática a través de un enfoque detallado en los apéndices, que abordan temas tan diversos como la inducción matemática, los números complejos, el teorema fundamental del álgebra y la resolución de ecuaciones bajo radicales. Aunque pueden parecer temas adicionales, estos apéndices son esenciales para comprender completamente las teorías matemáticas más avanzadas.

En primer lugar, el apéndice sobre los principios de inducción proporciona una herramienta útil para demostrar la validez de afirmaciones matemáticas a través del razonamiento por paso. Esta técnica es esencial para demostrar teoremas matemáticos complejos y se aplica en una variedad de campos dentro de la matemática.

El apéndice sobre los números complejos proporciona una introducción clara y concisa a este tema fundamental en matemáticas, brindando al lector las herramientas necesarias para comprender y aplicar conceptos relacionados con los números complejos a lo largo del texto principal.

El teorema fundamental del álgebra es un resultado central en el álgebra moderna y establece que cualquier ecuación polinómica tiene una solución compleja. Este teorema tiene aplicaciones en muchos campos relacionados con la ciencia y la ingeniería.

El apéndice sobre la resolución de ecuaciones bajo radicales presenta una técnica para simplificar ecuaciones algebraicas para que sean más manejables y fáciles de resolver. Aunque puede parecer un tema práctico y mundano, esta técnica es esencial para la resolución de problemas en álgebra y cálculo.

Recalco que esta obra son notas del curso de Álgebra Lineal del periodo 2024/1 y 2024/2, por lo que recomiendo consultar la bibliografía proporcionada en este trabajo. La presente obra consta de todo el contenido que adjudicó el Lic. Luis Alfonso Godinez Contreras, quien imparte clases en la Escuela Superior de Física y Matemáticas (ESFM) en el Instituto Politécnico Nacional (IPN). Hice el mayor esfuerzo en estructurar el libro a base de proposiciones, ejercicios, notas, etc.; y agregué y modifiqué algunas definiciones, a fin de lograr una mayor comprensión sin llegar a tener alguna ambigüedad.

Cabe mencionar que este libro fue mecanografiado por Marco Antonio Molina Mendoza, estudiante de la Escuela Superior de Física y Matemáticas. Sigo trabajando para corregir errores y/o mejorar este trabajo, por lo que está en constante cambio. Recomiendo visitar \href{https://linktr.ee/biblioteca_esfm}{\textbf{Biblioteca ESFM}}, un proyecto personal que compila apuntes actualizados. Allí podrán acceder a la última edición del documento.\marginElement{
\begin{center}
    \begin{tikzpicture}
        \node[fill=white] at (0,0) {\hypersetup{hidelinks}\qrcode[hyperlink, height=0.8\linewidth]{https://linktr.ee/biblioteca_esfm}};
    \end{tikzpicture}
\end{center}
}

Algunos cambios han sido resultado de los comentarios de mis amigos y estudiantes de la Escuela Superior de Física y Matemáticas. Estas son algunas de las muchas mejoras que he incorporado.
\begin{itemize}
    \item Aspecto estético mejorado: He trabajado arduamente en mejorar el aspecto visual del libro. Incorpore nuevas fuentes y estilos para que sea más fácil de leer y asimilar los conceptos.
    \item Errores ortográficos y matemáticos corregidos: He realizado una revisión minuciosa para garantizar la precisión en todos los aspectos del contenido. Los errores ortográficos y matemáticos se han corregido para ofrecer una experiencia de aprendizaje fluida y confiable.
    \item Imágenes actualizadas: He renovado todas las imágenes y gráficos del libro para asegurarme de que sean claros, informativos y visualmente atractivos. Las nuevas ilustraciones ayudarán a comprender mejor los conceptos y aplicaciones del álgebra lineal en el mundo real.\infoBulle{Las imágenes que se presentan en este libro están hechas con \texttt{TikZ}, una herramienta para la producción de gráficos vectoriales. Dichas imágenes son de elaboración propia y son de dominio público.}
    \item Ejemplos adicionales: He añadido más ejemplos paso a paso y problemas resueltos para reforzar el entendimiento de los temas discutidos.
    %\item Recuadros de resumen y notas destacadas: He agregado recuadros de resumen al final de cada sección para recapitular los puntos clave. Además, he incluido notas destacadas para enfatizar conceptos importantes y recordatorios útiles a lo largo del libro.
\end{itemize}\newpage\noindent
Además, quiero destacar que este libro, disponible en formato PDF, ofrece una funcionalidad adicional a través de tres hipervínculos de gran utilidad.
\begin{itemize}
    \item Números de páginas: Se ubican en la parte superior derecha e izquierda, dependiendo de si es una página par o impar. Al hacer clic, te redirigirán al índice del libro.
    \item Nombres de secciones en el encabezado: Se localizan en la parte superior derecha de cada página impar. Al hacer clic, te llevarán al inicio de la respectiva sección
    \item Nombres de capítulos en el encabezado: Se encuentran en la parte superior izquierda de cada página par. Al hacer clic, te dirigirán al inicio del correspondiente capítulo.
\end{itemize}

En la elaboración de este libro, he optado por una presentación que refleje la esencia misma de la disciplina: precisión, claridad y elegancia matemática. Con este propósito en mente, he desarrollado una plantilla en LaTeX que resalta la pureza del contenido, utilizando únicamente el contraste entre el blanco y el negro para enfocar la atención en los conceptos fundamentales que exploraremos a lo largo de estas páginas.

La elección de un diseño en blanco y negro no es solo estética, sino una decisión que busca facilitar la comprensión del material para los lectores. Al eliminar cualquier distracción visual que pueda surgir del uso de colores llamativos, me he enfocado en ofrecer una experiencia de lectura que permita una inmersión total en el mundo del Álgebra Lineal.

Las ecuaciones, teoremas y demostraciones, que constituyen el núcleo de este texto, se presentan de manera clara y legible, sin interferencias visuales. La tipografía se ha seleccionado con cuidado para garantizar una lectura fluida y agradable, mientras que las figuras y gráficos han sido diseñados con precisión para complementar y reforzar los conceptos expuestos en el texto.

Entiendo la importancia de la legibilidad tanto en la versión impresa como en la digital, por lo que he dedicado especial atención a la adaptabilidad de mi plantilla. Ya sea en un libro físico o en un documento electrónico, los lectores encontrarán una presentación coherente y accesible que facilitará su inmersión en los temas tratados.

Con una presentación clara y concisa de los temas, busco fomentar la apreciación de la belleza y utilidad de esta rama de las matemáticas. Se aspira a que los lectores adquieran las habilidades y conocimientos necesarios para resolver problemas complejos y se sientan motivados a explorar aplicaciones más avanzadas del álgebra lineal en su futuro académico y profesional.

Con gratitud y humildad, me complace presentar este volumen a la comunidad académica y a todos aquellos que tienen interés en explorar esta área de las matemáticas. Espero sinceramente que su lectura les resulte tan enriquecedora y reveladora como ha sido para mí su elaboración.

\begin{flushright}
    Marco Antonio Molina Mendoza\\ 
    México, \today\\ 
\end{flushright}