\chapter{NÚMEROS COMPLEJOS}\label{chap:numeros-complejos}
\printchaptertableofcontents

\section{El conjunto de los números complejos}

Los números complejos son una extensión del conjunto de los números reales. A medida que exploramos las complejidades del álgebra, nos encontramos con esta rama que añade una nueva dimensión a nuestro entendimiento numérico. La unidad imaginaria, denotada como $i$, desempeña un papel central en este contexto, ya que nos lleva más allá de las limitaciones de los números reales.

En el siglo XVI, los matemáticos se enfrentaron al problema de encontrar soluciones a ecuaciones cuadráticas que involucraban raíces cuadradas negativas. Por ejemplo, la ecuación \(x^2 + 1 = 0\) no tiene soluciones en los números reales, ya que no hay ningún número real cuyo cuadrado sea $-1$.

La historia de los números complejos es rica y tiene sus raíces en la necesidad de encontrar soluciones a ecuaciones aparentemente insolubles en el ámbito de los números reales. El concepto de números complejos se desarrolló a lo largo de varios siglos, con contribuciones significativas de matemáticos de diversas culturas. Aquí se mencionan algunas:
\begin{itemize}
    \item Cardano y la fórmula Cubica: En el siglo XVI, el matemático italiano Gerolamo Cardano desarrolló la fórmula para encontrar soluciones a ecuaciones cúbicas. Esta fórmula implicaba tomar raíces cuadradas de números negativos, aunque Cardano mismo no comprendía completamente las implicaciones de este proceso. En el \hyperref[sec:radical]{Apéndice D} se habla más detalladamente sobre la historia de la ecuación de segundo y tercer grado.
    \item Bombelli y la aceptación de lo “imaginario”: Rafael Bombelli, a finales del siglo XVI, fue uno de los primeros en tratar con números imaginarios de manera sistemática. En su obra “L'Algebra”, introdujo términos imaginarios para representar las raíces cuadradas de números negativos. Aunque no siempre entendía completamente la naturaleza de estos números, los utilizaba de manera efectiva.
    \item Descartes y la coordenada compleja: En el siglo XVII, René Descartes introdujo la representación geométrica de números complejos en un plano bidimensional, asociando la parte real con el eje horizontal y la parte imaginaria con el eje vertical. Esta representación facilitó la visualización de las operaciones con números complejos.
    \item Euler y la identidad Exponencial: En el siglo XVIII, Leonhard Euler desempeñó un papel crucial al establecer la forma polar de los números complejos y la famosa identidad de Euler
    \[e^{i\theta} = \cos(\theta) + i\sin(\theta).\]
    Esto unificó las representaciones rectangular y polar de los números complejos, proporcionando una herramienta poderosa para manipularlos.
\end{itemize}

En resumen, la historia de los números complejos es una narrativa fascinante que involucra la superación de obstáculos conceptuales y la evolución de la comprensión matemática. Desde sus raíces en la resolución de ecuaciones cuadráticas hasta su papel central en la física teórica moderna, los números complejos han demostrado ser una herramienta poderosa y versátil en el arsenal matemático.

\begin{definition}\label{definicion:definiciondecomplejo}
    Un número complejo $z$ es una pareja ordenada de $\RR$. El conjunto de números complejos lo denotaremos por $\CC$, donde
    $$\CC=\left\{ (a,  b) \mid a,  b \in \RR \right\} .$$
\end{definition}

\begin{definition}
    Sean $z$, $w \in \CC$, con $z=(a,  b)$ y $w=(c,  d)$. Decimos que dos números complejos $z$ y $w$ son iguales, y escribiremos $z=w$, si $a=c$ y $b=d$.
\end{definition}

\begin{definition}
    Sea $z \in \CC$, con $z=a+bi$. Definimos:
    \begin{enumerate}[label=\roman*.]
        \item La parte real de $z$, denotada por $\operatorname{Re}  (z)$, como $\operatorname{Re} (z)=a$.
        \item La parte imaginaria de $z$, denotada por $\operatorname{Im} (z)$, como $\operatorname{Im} (z)=b$.
    \end{enumerate}
\end{definition}

\section{Operaciones con números complejos}

El conjunto de los números complejos se caracteriza por la presencia de dos operaciones fundamentales: la suma y el producto. Para definir la suma de dos números complejos $z$ y $w$, representados como $z=(a,  b)$ y $w=(c,  d)$ en el conjunto $\CC$, utilizamos la siguiente expresión:
$$z + w=(a,  b)+(c,  d)=(a+c,  b+d).$$
Asimismo, para definir el producto de estos números complejos $z$ y $w$, expresados como $z=(a,  b)$ y $w=(c,  d)$ en el conjunto $\CC$, empleamos la siguiente fórmula:
$$zw=(a,  b)(c,  d)=(ac-bd,  ad+bc).$$
Es importante destacar que la suma cumple con las siguientes propiedades:\newpage
\begin{enumerate}[label=A\arabic*.]
    \item Cerradura: Si $z$, $w \in \CC$, entonces $z+w \in \CC$. En efecto: Sean $z$, $w \in \CC$ con $z=(a,  b)$ y $w=(c,  d)$, entonces
    \begin{align*}
        z+w & = (a,  b)+(c,  d) \\
        & = (a+c,  b+d) \in \CC.
    \end{align*}
    \item Asociatividad: Si $z$, $w$, $y \in \CC$, entonces
    $$(z+w)+y=z+(w+y).$$
    En efecto: Sean $z$, $w \in \CC$ con $z=(a,  b)$, $w=(c,  d)$ e $y=(e,  f)$. Entonces
    \begin{align*}
        (z+w)+y &=\{(a,  b)+(c,  d) \} +(c,  f) \\
        &=(a+c,  b+d)+(e,  f) \\
        &=\big( (a+c)+e,  (b+d)+f \big) \\
        &=\big( a+(c+e),  b+(d+f) \big) \\
        &=(a,  b)+(c+e,  d+f) \\
        &=(a,  b)+ \big( (c,  d)+(e,  f) \big) \\
        &=z+(w+y).
    \end{align*}
    \item Conmutatividad: Si $z$, $w \in \CC$, entonces
    $$z+w=w+z.$$
    En efecto: Sean $z$, $w \in \CC$ tal que $z=(a,  b)$ y $w=(c,  d)$, entonces
    \begin{align*}
        z+w &=(a,  b)+(c,  d) \\
        &=(a+c,  b+d) \\
        &=(c+a,  d+b) \\
        &=(c,  d)+(a,  b) \\
        &=w+z.
    \end{align*}
    \item Elemento neutro: Existe un único número complejo, denotado por $0$, que satisface
    $$z+0=z, \; \forall z \in \CC.$$
    En efecto: Sea $z \in \CC$, con $z=(a,  b)$.
    
    \textbf{Existencia:} Sea $w=(0,  0)$, entonces
    \begin{align*}
        z+w & = (a,  b)+(0,  0) \\
        & = (a+0,  b+0) \\
        & = (a,  b) \\
        & = z.
    \end{align*}
    Por lo que $z+w=z$
    
    \textbf{Unicidad:} Sea $v \in \CC$ tal que
    $$z+v=z, \; \forall z \in \CC,$$
    en particular $z=w$, tenemos que $w+v=w$. Pero $z+w=z$, $\forall z \in \CC$, en particular para $z=v$, tenemos $v+w=v$. Así que
    \begin{align*}
        v & = v+w \\
        & = w+v \\
        & = w.
    \end{align*}
    Por lo tanto $v=w$.\newpage
    \item Inverso aditivo: Dado $z \in \CC$, existe un único número complejo, denotado por $-z$ que satisface
    $$z+(-z)=0.$$
    En efecto: Sea $z \in \CC$, con $z=(a,  b)$.
    
    \textbf{Existencia:} Sea $w=(-a,  -b)$, entonces
    \begin{align*}
        z+w &=(a,  b)+(-a,  -b) \\
        &=\big( a+(-a),  b+(-b) \big) \\
        &=(0,  0) \\
        & = 0.
    \end{align*}
    \textbf{Unicidad:} Sea $v \in \CC$ tal que $z+v=0$. Tenemos
    \begin{align*}
        w &=w+0 \\
        &=w+(z+v) \\
        &=(w+z)+v \\
        &=0+v \\
        &=v.
    \end{align*}
    Por lo tanto $w=v$.
\end{enumerate}
El producto satisface:
\begin{enumerate}[resume,label=A\arabic*.]
    \item Cerradura: Sea $z$, $w \in \CC$, entonces $zw \in \CC$. En efecto: Sea $z$, $w \in \CC$ con $z=(a,  b)$ y $w=(c, \ d)$, entonces
    \begin{align*}
        zw & = (a,  b) (c,  d) \\
        & = (ac-bd,  ad+bc) \in \CC.
    \end{align*}
    \item Asociatividad: Si $z$, $w$, $y \in \CC$, entonces
    $$(zw)y=z(wy).$$
    En efecto: Sean $z$, $w$, $y \in \CC$ con $z=(a,  b)$, $w=(c,  d)$ e $y=(e,  f)$. Entonces
    \begin{align*}
        (zw)y &=\{ (a,  b)(c,  d) \} (e,  f) \\
        &=(ac-bd,  ad+bc)(e,  f) \\
        &=\big( (ac-bd)(e)-(ad+bc)(f),  (ac-bd)(f)+(ad+bc)(e) \big) \\
        &=(ace-bde-adf-bcf,  acf-bdf+ade+bce) \\
        &=(ace-adf-bcf-bde,  acf+ade+bce-bdf) \\
        &=\big( a(ce-df)-b(cf+de),  a(cf+de)+b(ce-df) \big) \\
        &=(a,  b)(ce-df,  cf+de) \\
        &=(a,  b)\big( (c,  d)(e,  f) \big) \\
        &=z(wy).
    \end{align*}
    \item Conmutatividad: Si $z$, $w \in \CC$. Entonces
    $$zw=wz.$$
    En efecto: Sean $z$, $w \in \CC$ con $z=(a,  b)$ y $w=(c,  d)$, entonces
    \begin{align*}
        zw &=(a,  b)(c,  d) \\
        &=(ac-bd,  ad+bc) \\
        &=(ca-db,  cb+da) \\
        &=(c,  d)(a,  b) \\
        &=wz.
    \end{align*}
    \item Elemento identidad: Existe un único complejo, denotado por 1 que satisface
    $$1 \cdot z=z, \; \forall z \in \CC.$$
    En efecto: Sea $z \in \CC$ con $z=(a,  b)$.
    
    \textbf{Existencia:} Sea $w=(1,  0)$, entonces
    \begin{align*}
        wz & = (1,  0)(a,  b) \\
        & = (1a-0b,  1b+0a) \\
        & = (a,  b) \\
        & = z.
    \end{align*}
    
    \textbf{Unicidad:} Sea $v \in \CC$ tal que
    $$vz=z, \; \forall z \in \CC,$$
    en particular para $z=w$, entonces $vw=w$. Como $wz=z$, $\forall z \in \CC$, en particular $z=v$, entonces $wv=v$. Así que
    \begin{align*}
        v & = wv \\
        & = vw \\
        & = w.
    \end{align*}
    Por tanto, $v=w$.
    \item Inverso multiplicativo: Dado $z \in \CC$, con $z \neq 0$, existe un único complejo tal que
    $$z \cdot z^{-1}=1.$$
    En efecto: Sea $z \in \CC$, con $z=(a,  b)$ y $z \neq 0$, entonces $a \neq 0$ o $b \neq 0$. Deseamos encontrar $w \in \CC$ tal que $zw=(1,  0)$. Expresemos $w=(x,  y)$, entonces
    \begin{align*}
        (1,  0) &=zw \\
        &=(a,  b)(x,  y) \\
        &=(ax-by,  ay+bx).
    \end{align*}
    Luego $ax-by=1$ y $ay+bx=0$. Es decir
    $$\left\{ \begin{array}{rl}
        ax-by= & \!\!\!\! 1 \\
        ay+bx= & \!\!\!\! 0
    \end{array}\right.$$
    \begin{enumerate}[label=\roman*.]
        \item Si $a=0$, entonces $b \neq 0$. Luego $by=1$, así que $\displaystyle y=-\frac{1}{b}$, entonces $bx=0$, así que $x=0$.
        \item Si $b=0$, entonces $a \neq 0$. Luego $ax=1$, así que $\displaystyle x=\frac{1}{a}$, entonces $ay=0$, entonces $y=0$.
        \item Supongamos que $a$, $b \neq 0$. Entonces $a^2+b^2 \neq 0$, luego
        $$\begin{array}{rl}
            a^2x-aby= & \!\!\!\! a \\
            aby+b^2x= & \!\!\!\! 0
        \end{array} = \begin{array}{rl}
            a^2x-aby= & \!\!\!\! a \\
            b^2x+aby= & \!\!\!\! 0
        \end{array} \Longrightarrow x \left(a^2+b^2 \right)= a$$
        Por lo tanto, $\displaystyle x=\frac{a}{a^2+b^2}$. Por otro lado
        $$\begin{array}{rl}
            -bax+b^2y= & \!\!\!\! -b \\
            a^2y+abx= & \!\!\!\! 0
        \end{array} =
        \begin{array}{rl}
            -bax+b^2y= & \!\!\!\! -b \\
            bax+a^2y= & \!\!\!\! 0
        \end{array} \Longrightarrow y \left( a^2+b^2 \right)=-b$$
        Por lo tanto, $\displaystyle y=-\frac{b}{a^2+b^2}$. Por tanto, $\displaystyle z^{-1}=\left( \frac{a}{a^2+b^2},  \frac{-b}{a^2+b^2} \right)$.
    \end{enumerate}
\end{enumerate}\newpage
La adición y el producto se relacionan con la siguiente propiedad:
\begin{enumerate}[resume,label=A\arabic*.]
    \item Distributividad: Sean $z$, $w$, $y \in \CC$, entonces
    $$z(x+y) = zx + zy.$$
    En efecto: Se deja como ejercicio al lector.
\end{enumerate}

\begin{observation}
    Sea $X=\{(a,  0) \mid a \in \RR \}$. Sea $f: \RR \longrightarrow X$ dada por $f(a)=(a,  0)$, para todo $a \in \RR$. Sean $a,  b \in \RR$. Entonces
    \begin{align*}
        f(a+b) &=(a+b,  0) \\
        & =(a,  0)+(b,  0) \\
        &=f(a)+f(b),
    \end{align*}
    y
    \begin{align*}
        f(a) f(b) &=(a,  0)(b,  0) \\
        & =(a b - 0 \cdot 0,  a \cdot 0 + 0 \cdot b) \\
        & =(a b,  0) \\
        & =f(a b).
    \end{align*}
    \begin{enumerate}[label=\roman*.]
        \item Para demostrar que $f$ es una función inyectiva, consideremos $a, b \in \RR$ tal que $f(a)=f(b)$. Esto implica que $(a,  0)=(b,  0)$, lo que a su vez conduce a $a=b$. Por lo tanto, concluimos que $f$ es inyectiva.
        \item Para demostrar la sobreyectividad de $f$, tomemos $(a, 0) \in X$. Entonces, se cumple que $f(a)=(a,  0)$.
        \item Dado que hemos demostrado tanto la inyectividad como la sobreyectividad de $f$, podemos concluir que $f$ es una función biyectiva.
    \end{enumerate}
    Basándonos en lo expuesto anteriormente, podemos afirmar que $X$ es isomorfo al conjunto de los números reales, representado como $\RR$. Esto implica que si consideramos un elemento $a$ perteneciente a los números reales, podemos asociarlo con $(a,  0)$. De manera abreviada y abusando de la notación, podemos expresar $a=(a,  0)$. Por lo tanto, podemos establecer las siguientes identidades:
    $$1=(1,  0) \quad \text{ y } \quad 0=(0,  0).$$
\end{observation}

\begin{observation}
    Tomemos $i=(0,1)$. Observemos que
    \begin{align*}
        i^2 &=(0,  1)(0,  1) \\
        &=\big( 0(0)-1(1),  0(1)+1(0)\big)\\
        &=(-1,  0)\\
        &=-1.
    \end{align*}
    Además si $b \in \RR$, entonces
    \begin{align*}
        b i & =(b,  0)(0,  1) \\
        & =(b \cdot 0-0 \cdot 1,  b \cdot 1+0 \cdot 0) \\
        & =(0,  b).
    \end{align*}
    Por lo tanto, si $z \in \CC$, con $z=(a,  b)$, entonces:\infoBulle{A la expresión $a+bi$ se le llama forma binómica donde $a$ es la parte real y $b$ es la parte imaginaria. Esta notación facilita la manipulación algebraica de los números complejos.}
    \begin{align*}
        z &=(a,  b) \\
        & =(a,  0)+(0,  b) \\
        & =(a, 0)+(b, 0)(0,1) \\
        & =a+b i.
    \end{align*}
\end{observation}

\newpage

\begin{observation}
    Sean $z$, $w \in \CC$ con $z=a+bi$ y $w=c+di$, entonces
    \begin{align*}
        z+w &=(a+bi)+(c+di) \\
        &=(a+c)+(bi+di) \\
        &=(a+c)+(b+d)i,
    \end{align*}
    y
    \begin{align*}
        zw &=(a+bi)(c+di) \\
        &=ac+adi+bci+bidi \\
        &=ac+(ad+bc)i-bd \\
        &=(ac-bd)+(ad+bc)i.
    \end{align*}
    Si $w \neq 0$
    \begin{align*}
        w^{-1} &=\frac{1}{w} \\
        &=\frac{1}{c+di} \\
        &=\frac{1}{c+di} \cdot \frac{c-di}{c-di} \\
        &=\frac{c-di}{(c+di)(c-di)} \\
        &=\frac{c}{c^2+d^2} - \frac{d}{c^2+d^2} i.
    \end{align*}
    Por tanto, $\displaystyle w^{-1}=\frac{c}{c^2+d^2} - \frac{d}{c^2+d^2} i$.
\end{observation}

\begin{definition}
    Sean $z$, $w \in \CC$, definimos
    \begin{enumerate}[label=\roman*.]
        \item $z-w=z+(-w)$.
        \item $\displaystyle \frac{z}{w}=zw^{-1}$, con $w \neq 0$.
    \end{enumerate}
\end{definition}

\begin{observation}
    Sean $z$, $w \in \CC$ con $z=a+bi$ y $w=c+di$, entonces
    \begin{align*}
        z-w &=(a+bi)-(c+di) \\
        &=(a-c)+(b-d)i.
    \end{align*}
    Si $w \neq 0$, entonces
    \begin{align*}
        \frac{z}{w} &= \frac{a+bi}{c+di} \\
        &= \frac{a+bi}{c+di} \cdot \frac{c-di}{c-di} \\
        &= \frac{ac+bd+(-ad+cb)i}{c^2+d^2} \\
        & =\frac{ac+bd}{c^2+d^2} + \frac{-ad+cb}{c^2+d^2} i.
    \end{align*}
\end{observation}

\begin{proposition}
    Sean $z$, $z_1$, $z_2$, $z_3 \in \CC$, entonces
    \begin{enumerate}[label=\roman*.]
        \item $z \cdot 0=0$.
        \item $-z=(-1)z$.
        \item $-(-z)=z$.
        \item $(-z_1)(z_2)=z_1(-z_2)=-(z_1z_2)$.
        \item $(-z_1)(-z_2)=z_1z_2$.
        \item $z_1(z_2-z_3)=z_1z_2-z_1z_3$.
    \end{enumerate}
    \demostracion Se dejan como ejercicio al lector.
\end{proposition}

\section{El conjugado de un número complejo}

\begin{definition}
    Sea $z \in \CC$, con $z=a+bi$. Definimos el conjugado de $z$, denotado por $\overline{z}$, como $\overline{z} =a-bi$.
\end{definition}

\begin{proposition}\label{conjugado-real}
    Sea $z \in \CC$. Entonces $z \in \RR$ si y solo si $\overline{z}=z$. \\
    \demostracion Expresemos $z=a+bi$. Supongamos que $z \in \RR$, entonces $z=a+0i$, luego
    \begin{align*}
        \overline{z} &=a-0i \\
        & = a \\
        & = z.
    \end{align*}
    Por lo tanto $\overline{z}=z$. Supongamos que $\overline{z}=z$, entonces $a-bi=a+bi$, luego $a=a$ y $-b=b$, así que $b=0$ y por lo tanto $z=a \in \RR$.
\end{proposition}

\begin{proposition}
    Si $z$, $w \in \CC$, entonces $\overline{z+w}=\overline{z} +\overline{w}$. \\
    \demostracion Sean $z=a+bi$ y $w=c+di$, con $z$, $w \in \CC$, entonces $z+w=(a+c)+(b+d)i$, así pues
    \begin{align*}
        \overline{z+w} & =(a+c)-(b+d)i \\
        &=a+c-bi-di \\
        &=(a-bi)+(c-di) \\
        &=\overline{z} + \overline{w}.
    \end{align*}
    Por lo tanto $\overline{z+w}=\overline{z} + \overline{w}$.
\end{proposition}

\begin{proposition}
    Si $z \in \CC$, entonces $\overline{-z}=- \overline{z}$. \\
    \demostracion Sea $z \in \CC$ tal que $z=a+bi$, entonces $-z=-a-bi$, luego
    \begin{align*}
        \overline{-z} &=-a+bi \\
        & =-(a-bi) \\
        & =- \overline{z}.
    \end{align*}
    Por tanto, $\overline{-z}=- \overline{z}$.
\end{proposition}

\begin{proposition}
    Si $z$, $w \in \CC$, entonces $\overline{z-w}=\overline{z} - \overline{w}$. \\
    \demostracion Sean $z$, $w \in \CC$ tal que $z=a+bi$ y $w=c+di$, entonces
    \begin{align*}
        \overline{z-w} &=\overline{z+(-w)} \\
        &=\overline{z} + \overline{(-w)} \\
        &=\overline{z} + (- \overline{w}) \\
        &=\overline{z} - \overline{w}.
    \end{align*}
    Por tanto, $\overline{z-w}=\overline{z} - \overline{w}$.
\end{proposition}

\begin{proposition}
    Si $z$, $w \in \CC$, entonces $\overline{zw}=\overline{z} \cdot \overline{w}$. \\
    \demostracion Sean $z$, $w \in \CC$ tal que $z=a+bi$ y $w=c+di$, entonces $\overline{z}=a-bi$ y $\overline{w}=c-di$. Así pues
    \begin{align*}
        z \cdot w &=(ac-bd)+(ad+bc)i \\
        \overline{z \cdot w} &=(ac-bd)-(as+bc)i,
    \end{align*}
    en consecuencia
    \begin{align*}
        \overline{z} \cdot \overline{w} &=(a-bi)(c-di) \\
        &=\big( ac-(-b)(-d) \big) +\big( a(-d)+(-b)c \big) i \\
        &=(ac-bd)+(-ad-bc)i \\
        &=(ac-bd)-(ad+bc)i.
    \end{align*}
    Por lo tanto, $\overline{zw}=\overline{z} \cdot \overline{w}$.
\end{proposition}

\begin{proposition}
    Si $z \in \CC$, con $z \neq 0$, entonces $\overline{z^{-1}}=\left( \overline{z} \right)^{-1}$. \\
    \demostracion Sabemos que $z \cdot z^{-1}=1$, entonces
    \begin{align*}
        1 &=\overline{1} \\
        & =\overline{z \cdot z^{-1}} \\
        & =\overline{z} \cdot \overline{z^{-1}},
    \end{align*}
    por lo tanto $\overline{z^{-1}}=\left( \overline{z} \right)^{-1}$.
\end{proposition}

\begin{proposition}
    Si $z$, $w \in \CC$, con $w \neq 0$, entonces $\displaystyle \overline{\left( \frac{z}{w} \right)}=\frac{\overline{z}}{\overline{w}}$. \\
    \demostracion Sean $z$, $w \in \CC$, con $w \neq 0$. Entonces
    \begin{align*}
        \overline{\left( \frac{z}{w} \right)} &= \overline{z \cdot w^{-1}} \\
        & =\overline{z} \cdot \overline{w^{-1}} \\
        & =\overline{z} (\overline{w})^{-1} \\
        & =\frac{\overline{z}}{\overline{w}}.
    \end{align*}
    Por lo tanto, $\displaystyle \overline{\left( \frac{z}{w} \right)}=\frac{\overline{z}}{\overline{w}}$.
\end{proposition}

\begin{proposition}
    Si $z \in \CC$, entonces $z+\overline{z}= 2  \operatorname{Re}(z)$. \\
    \demostracion Sea $z \in \CC$ tal que $z=a+bi$, entonces
    \begin{align*}
        z+\overline{z} &=(a+bi)+(a-bi) \\
        & =2a \\
        & =2  \operatorname{Re}(z).
    \end{align*}
    Por lo tanto, $z+\overline{z}= 2  \operatorname{Re}(z)$.
\end{proposition}

\begin{proposition}
    Si $z \in \CC$, entonces $z-\overline{z}= 2  \operatorname{Im}(z)  i$. \\
    \demostracion Sea $z \in \CC$ tal que $z=a+bi$, entonces
    \begin{align*}
        z-\overline{z} &=(a+bi)-(a-bi) \\
        & =2bi \\
        & =2  \operatorname{Im}(z)  i
    \end{align*}
    Por tanto, $z-\overline{z}= 2  \operatorname{Im}(z)  i$.
\end{proposition}

\begin{proposition}
    Si $z \in \CC$, entonces $\overline{\overline{z}}=z$. \\
    \demostracion Sea $z \in \CC$ con $z=a+bi$, entonces $\overline{z}=a-bi$ y $\overline{\overline{z}}=a+bi$. Por tanto, $ \overline{\overline{z}}=z$.
\end{proposition}

\section{El módulo de un número complejo}

\begin{definition}
    Sea $z=a+bi$ un número complejo, definimos el módulo (o también conocido como valor absoluto), denotado por $|z|$, como la raíz cuadrada del número real $a^2+b^2$, es decir,
    $$|z|=\sqrt{a^2+b^2}.$$
\end{definition}

\begin{proposition}
    Si $z \in \CC$, entonces
    \begin{enumerate}[label=\roman*.]
        \item $|z| \geq 0$.
        \item $|\overline{z}|=|z|$.
        \item $|z|=0$ si y solo si $z=0$.
        \item $|z|^2=z \overline{z}$.
    \end{enumerate}
    \demostracion
    \begin{enumerate}[label=\roman*.]
        \item Sea $z \in \CC$, con $z=a+bi$, entonces
        $$a^2+b^2 \geq 0.$$
        Luego $|z|=\sqrt{a^2+b^2} \geq 0$.
        \item Sea $z \in \CC$, con $z=a+bi$, tenemos
        \begin{align*}
            |\overline{z}| &=|a-bi| \\
            &=\sqrt{a^2+(-b)^2} \\
            &=\sqrt{a^2+b^2} \\
            &=|z|.
        \end{align*}
        Por lo tanto, $|\overline{z}|=|z|$.
        \item Sea $z \in \CC$, con $z=a+bi$, entonces
        \begin{align*}
            |z|=0 & \Longleftrightarrow \sqrt{a^2+b^2}=0 \\
            & \Longleftrightarrow a^2+b^2=0 \\
            & \Longleftrightarrow a^2=0 \text{ y }  b^2=0 \\
            & \Longleftrightarrow a=0 \text{ y }  b=0 \\
            & \Longleftrightarrow z=0.
        \end{align*}
        \item Sea $z \in \CC$, con $z=a+bi$, entonces
        \begin{align*}
            z \overline{z} & = (a+bi)(a-bi) \\
            & = a^2 - (bi)^2 \\
            & = a^2 + b^2 \\
            & = \left| a^2 + b^2 \right| \\
            & = \sqrt{\left(a^2+b^2\right)^2} \\
            & = \left(\sqrt{a^2+b^2}\right)^2 \\
            & = |z|^2.
        \end{align*}
    \end{enumerate}
\end{proposition}

\begin{theorem}
    Si $z$, $w \in \CC$, entonces $|zw|=|z||w|$. \\
    \demostracion Sea $z \in \CC$, con $z=a+bi$. Entonces
    \begin{align*}
        |zw|^2 &=(zw)\left( \overline{zw} \right) \\
        &=(zw)(\overline{z} \overline{w}) \\
        &=(z \overline{z})(w \overline{w}) \\
        &=|z|^2 |w|^2 \\
        &=(|z| |w|)^2.
    \end{align*}
    Luego $\sqrt{|zw|^2}=\sqrt{(|z| |w|)^2}$, así que
    $$\big| |zw| \big| =\big| |z| |w| \big|.$$
    Por lo tanto,
    $$|zw|=|z| |w|.$$
\end{theorem}

\newpage

\begin{proposition}
    Si $z \in \CC$, con $z \neq 0$, entonces $\left| z^{-1} \right| =|z|^{-1}$. \\
    \demostracion Tenemos que $z \cdot z^{-1}=1$. Luego
    \begin{align*}
        1 &=|1| \\
        &=\left| z \cdot z^{-1} \right| \\
        &=|z| \left| z^{-1} \right|.
    \end{align*}
    Así que
    $$|z| \left| z^{-1} \right| = 1.$$
    Por lo tanto
    $$|z|^{-1}=\left| z^{-1} \right|.$$
\end{proposition}

\begin{proposition}
    Si $z$, $w \in \CC$, con $w \neq 0$, entonces $\displaystyle \left| \frac{z}{w} \right| =\frac{|z|}{|w|}$. \\
    \demostracion Tenemos
    \begin{align*}
        \left| \frac{z}{w} \right| =\left| z \cdot w^{-1} \right|  &=|z| \left| w^{-1} \right| \\
        &=|z| \left| w \right| ^{-1} \\
        &=\frac{|z|}{|w|}.
    \end{align*}
    Por lo tanto,
    $$\left| \frac{z}{w} \right| =\frac{|z|}{|w|}.$$
\end{proposition}

\begin{proposition}
    Si $z \in \CC$, entonces
    \begin{enumerate}[label=\roman*.]
        \item $\operatorname{Re}(z) \leq |z|$. 
        \item $\operatorname{Im}(z) \leq |z|$.
    \end{enumerate}
    \demostracion Sea $z \in \CC$, con $z=a+bi$
    \begin{enumerate}[label=\roman*.]
        \item Como $b^2 \geq 0$, entonces $a^2+b^2 \geq a^2$, luego $\sqrt{a^2+b^2} \geq \sqrt{a^2}$, por lo que
        \begin{align*}
            |z| &= \sqrt{a^2+b^2} \\
            & \geq \sqrt{a^2} \\
            & =|a| \\
            & \geq a \\
            & =\operatorname{Re}(z).
        \end{align*}
        \item Como $a^2 \geq 0$, entonces $a^2+b^2 \geq b^2$, luego $\sqrt{a^2+b^2} \geq \sqrt{b^2}$, por lo que
        \begin{align*}
            |z| &= \sqrt{a^2+b^2} \\
            & \geq \sqrt{b^2} \\
            & =|b| \\
            & \geq b \\
            & =\operatorname{Im}(z).
        \end{align*}
    \end{enumerate}
\end{proposition}

\begin{theorem}
    Si $z$, $w \in \CC$, entonces
    $$|z+w| \leq |z|+|w|.$$\newpage
    \demostracion Sean $z$, $w \in \CC$, entonces
    \begin{align*}
        |z+w|^2 &=(z+w)\left( \overline{z+w} \right) \\
        &=(z+w)(\overline{z}+ \overline{w}) \\
        &=(z \overline{z})+(\overline{w} z)+(w \overline{z})+(w \overline{w}) \\
        &=|z|^2+z\overline{w}+\overline{z} \overline{\overline{w}}+|w|^2 \\
        &=|z|^2+z \overline{w}+\overline{z\overline{w}}+|w|^2 \\
        &=|z|^2+2  \operatorname{Re}(z \overline{w})+|w|^2 \\
        &\leq |z|^2 +2|z \overline{w}| +|w|^2 \\
        &\leq |z|^2 +2|z||\overline{w}|+|w|^2 \\
        &=|z|^2+2|z||w|+|w|^2 \\
        &\leq (|z|+|w|)^2.
    \end{align*}
    Luego
    $$|z+w|^2 \leq (|z|+|w|)^2.$$
    Por lo que
    $$\sqrt{|z+w|^2} \leq \sqrt{(|z|+|w|)^2}.$$
    Así que
    $$\big| |z+w| \big| \leq \big| |z|+|w| \big|.$$
    Por lo tanto,
    $$|z+w| \leq |z|+|w|.$$
\end{theorem}

\section{La representación geométrica de un número complejo}

Siguiendo la definición \ref{definicion:definiciondecomplejo}, podemos representar un número complejo en un plano cartesiano, llamado también plano complejo.
\begin{center}
    \begin{tikzpicture}[scale=0.9]
        \draw[thick,dash pattern=on 3pt off 3pt] (0,3) -- (3,3) -- (3,0);
        \draw[thick,dash pattern=on 3pt off 3pt] (3,0) -- (3,-3) -- (0,-3);
        \draw[thick,dash pattern=on 3pt off 3pt] (-3,0) -- (-3,3) -- (0,3);
        \draw[thick,dash pattern=on 3pt off 3pt] (-3,0) -- (-3,-3) -- (0,-3);
        
        \node at (0,5.3) {Eje imaginario};
        \node at (5.9,0) {Eje real};
        
        \draw[thick,-stealth] (0,0) -- (-5,0);
        \draw[thick,-stealth] (0,0) -- (5,0);
        \draw[thick,-stealth] (0,0) -- (0,5);
        \draw[thick,-stealth] (0,0) -- (0,-5);
        
        \filldraw (3,3) circle (2pt) node[right] {$z=a+bi$}; 
        \filldraw (3,-3) circle (2pt) node[right] {$\overline{z}=a-bi$};
        \filldraw (-3,3) circle (2pt) node[left] {$\overline{-z}=-a+bi$}; 
        \filldraw (-3,-3) circle (2pt) node[left] {$-z=-a-bi$}; 
        
        \filldraw (3,0) circle (2pt) node[below right] {$a$}; 
        \filldraw (-3,0) circle (2pt) node[below right] {$-a$}; 
        \filldraw (0,3) circle (2pt) node[above right] {$b$};
        \filldraw (0,-3) circle (2pt) node[below right] {$-b$}; 
    \end{tikzpicture}
\end{center}