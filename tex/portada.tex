\includepdf[pages=-]{GOOODINEZ.pdf}

\newpage

\symmetricalPage%

\begin{tikzpicture}[remember picture,overlay]%
      \coordinate (A) at ($(current page.north)!.4!(current page.north east)$);
      \coordinate (B) at (current page.east);
      \coordinate (C) at (current page.north east);

      \filldraw[black!50] (A) -- (B) -- (C) -- cycle;
      \filldraw ($(current page.north)!.4!(current page.north east)+(0.5,0)$) -- ($(current page.east)+(0,0.25)$) -- (C) -- cycle;
\end{tikzpicture}


\vspace{2cm}
\begin{center}
\scalebox{6.5}{\fontencoding{T1}\fontfamily{pag}\selectfont\textbf{ÁLGEBRA}}\\[10mm]
\scalebox{6.5}{\fontencoding{T1}\fontfamily{pag}\selectfont\textbf{LINEAL}}\\[2cm]

\scalebox{2}{\textbf{Luis Alfonso Godínez Contreras}}\\[4mm]
Escuela Superior de Física y Matemáticas\\[1.5cm]
\begin{center}
    \begin{tikzpicture}
        \node at (current page.center) {\includegraphics[width=0.3\textwidth]{Images/Portada/POLITECNICO.pdf}}; % Logo de institución
    \end{tikzpicture}
\end{center}
\vfill
México, 2024
\end{center}

\newpage
\,

\newpage

~\vfill\hfill\begin{minipage}[l]{0.5\textwidth}
     \itshape Quiero expresar mi agradecimiento al Lic. Luis Alfonso Godínez Contreras, quien aparte de haber sido mi profesor, me permitió recopilar el presente trabajo.
\end{minipage}

\restoregeometry%

\newpage
\,