\chapter*{BIBLIOGRAFÍA}
\addcontentsline{toc}{chapter}{BIBLIOGRAFÍA}

El presente trabajo se fundamenta en una amplia exploración del campo del álgebra lineal, abarcando sus principios fundamentales, aplicaciones y desarrollos contemporáneos. Los conceptos y teoremas presentados en esta obra se basan en una variedad de fuentes académicas y literarias que han contribuido al desarrollo y comprensión de esta disciplina matemática.

El estudio del álgebra lineal se nutre de una rica tradición de textos clásicos y contemporáneos que abordan la teoría y las aplicaciones de este campo. Las obras consultadas abarcan una amplia gama de enfoques, desde textos introductorios destinados a estudiantes principiantes hasta tratados avanzados dirigidos a investigadores y profesionales de las matemáticas y disciplinas afines.

La bibliografía incluye textos que exploran temas fundamentales como sistemas de ecuaciones lineales, vectores y espacios vectoriales, matrices y determinantes, así como transformaciones lineales y valores propios. Además, se pueden encontrar obras que profundizan en áreas especializadas como álgebra lineal numérica, álgebra lineal aplicada, y su intersección con campos como la física, la ingeniería, la informática y la inteligencia artificial.

Los autores cuyas obras han sido consultadas abarcan una amplia gama de expertos reconocidos en el campo del álgebra lineal, incluyendo académicos, investigadores y profesionales con una trayectoria destacada en el desarrollo y la aplicación de esta disciplina matemática.

La presente bibliografía no pretende ser exhaustiva, sino más bien servir como punto de partida para aquellos lectores interesados en explorar más a fondo los conceptos y aplicaciones del álgebra lineal. Se alienta a los lectores a consultar las obras citadas y a explorar la vasta literatura académica y científica disponible en este campo en constante evolución.

\section*{Fuente básica}

\begin{enumerate}[label={[\arabic*]}]
    \item Grossman, S. I. (2012). Álgebra Lineal (7a. ed.). México: Mc Grawl-Hill Interamerican.
\end{enumerate}

\section*{Fuentes de consulta}

\begin{enumerate}[resume,label={[\arabic*]}]
    \item Anton, H. (1991). Introducción al Álgebra Lineal (1a. ed.). México: Limusa.
    \item Herstein, I. N. y David J. Winter (1988). Álgebra Lineal y Teoría de Matrices. México: Iberoámerica.
    \item Grossman, S. I. (1988). Aplicaciones de Álgebra Lineal. Mexico: Iberoámerica.
    \item Fraleigth Beanregard (1990). Linear Algebra. USA: Addiwn-Wesley Publishing Company.
    \item Francis G. Florey (1979). Fundamentos de Álgebra Lineal y Aplicaciones. USA: Editorial Prentice Hall.
    \item Noble, Ben/Daniel, James W.; Álgebra Lineal Aplicada. Ed. Prentice Hall, 1989.
    \item Moore, John. T; Elements of Linear Algebra and Matrix Theory. Ed. Mc Graw Hill, 1968.
    \item Hoffman, Kenneth/Kunze, Ray; Álgebra Lineal. Ed. Prentce Hall, 1973.
    \item De Burgos, Juan: Algebra Lineal. Ed. Mc. Graw Hill, 1993.
    \item Fraleigh/Beauregard; Linear Algebra. Ed. Addison-Wesley, 1990.
    \item Voyevodin: Linear Algebra. Ed. Mir Moscú, 1980.
\end{enumerate}

\section*{Fuentes complementarias}

\begin{enumerate}[resume,label={[\arabic*]}]
    \item El Método de la Inducción Matemática, Sominskii, Editorial Limusa-Wiley, S. A. Primera Edición, México, 1970.
    \item ¿Qué son las Matemáticas?, Courant, Richard y Herbert Robbins, Editorial Fondo de Cultura Económica. México. Primera Edición en español 2002.
    \item Introducción al Cálculo, Kuratowski, Editorial Limusa-Wiley, S. A.
    \item Cálculo infinitesimal, Spivack, Michael, Editorial Reverté, S. A.
    \item An Introduction to Numerical Analysis, Atkinson, Kendall, Editorial John Wiley \& Sons., 1978.
    \item Visual Complex Analisis, T. Needham. New York: Oxford University, 1997.
    \item A History of Algebra, Springer Verlag, B. L. van der Waerden, New York, 1985.
    \item The Secret Formula, Toscano, F. Princeton University, 2020.
\end{enumerate}