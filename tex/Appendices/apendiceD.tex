\chapter[RESOLUCIÓN DE ECUACIONES BAJO RADICALES]{RESOLUCIÓN DE \\ ECUACIONES BAJO \\ RADICALES}\label{sec:radical}

De acuerdo con la versión de diferentes autores, mientras las ecuaciones generales de primer y segundo grados fueron resueltas desde la antigüedad, la ecuación general de tercer grado había resistido todos los esfuerzos de los matemáticos anteriores al italiano Scipio Del Ferro, quien finalmente logró resolverla a principios del siglo XVI, durante el renacimiento en Italia. De acuerdo a la costumbre de su tiempo, Del Ferro no publicó sus resultados, sino que se los comunicó a uno de sus discípulos, mismo que, tras la muerte de aquel, retó al gran matemático Tartaglia (1500-1557), también italiano, a que resolviera un cierto número de ecuaciones de tercer grado. Tartaglia aceptó el reto y, antes del plazo fijado en este, encontró un método para resolver cualquier ecuación cúbica de la forma \( x^3 + px + q = 0 \). Así como Del Ferro, Tartaglia no publicó su método; pero un profesor de física y matemática de Milán, Cardan (1501-1576), lo convenció de que se lo comunicara, bajo la promesa de mantenerlo en secreto. Cardan violó su promesa y publicó el método de Tartaglia en su trabajo \textit{Ars Magna} (El arte sumo) en 1545. Desde entonces, las fórmulas para resolver una ecuación de tercer grado se conocen como fórmulas de Cardan.

Poco después de la resolución de la ecuación cúbica, el también matemático italiano, alumno de Cardan, Ferrari (1522-1565) resolvió la ecuación general de cuarto grado.

Los procesos que aquí exponemos, para obtener las fórmulas para resolver las ecuaciones de tercer y cuarto grados, se sustentan en los trabajos de los matemáticos italianos mencionados, y se basan en transformaciones especiales y complicadas de la ecuación respectiva, que bien pueden parecer artificiales y accidentales, pero que ocurrieron en la intensa búsqueda de métodos para resolver dichas ecuaciones.

%\newpage

De la ecuación de primer grado, es decir
$$ax+b=0$$
solo diremos que su única solución es $\displaystyle x = -\frac{b}{a}$.

\section{La definición del discriminante}

\begin{definicion}{}{B.1.1}
    Consideremos la ecuación de grado $n \geq 2$ y coeficientes complejos dada por
    $$a_nx^n+\cdots +a_1x+a_0=0$$
    y sean $x_1, x_2, \dots, x_n$ sus raíces (no necesariamente distintas). Definimos el discriminante de esta ecuación como
    $$D=a_n^{2n-2}\prod_{1 \leq i < j \leq n}(x_i-x_j)^2.$$
\end{definicion}

En lugar de la ecuación
$$a_nx^n+\cdots +a_1x+a_0=0$$
de la definición anterior, podemos hablar del polinomio
$$f(x)=a_nx^n+\cdots +a_1x+a_0.$$
En este caso, se dice que
$$D=a_n^{2n-2}\prod _{1 \leq i < j \leq n}(x_i-x_j)^2$$
es el discriminante del polinomio $f(x)$.

\section{Ecuación de segundo grado}

Consideremos la ecuación general de segundo grado, donde $a, b, c \in \CC$ y $a \neq 0$, con la incógnita $x$:
\begin{equation}
    ax^2+bx+c=0. \label{ecuaciongrado2}
\end{equation}
Si $r$ es raíz de \eqref{ecuaciongrado2}, entonces
\begin{equation}
    ar^2+br+c=0. \label{ecuaciongrado2.1}
\end{equation}
Multiplicando ambos miembros de \eqref{ecuaciongrado2.1} por $4a$, obtenemos
\begin{equation*}
    4a^2r^2+4abr+4ac=0. \label{ecuaciongrado2.2}
\end{equation*}
Sumando $b^2-4ac$ en ambos miembro de \eqref{ecuaciongrado2.2}, obtenemos
$$4a^2r^2+4abr+b^2=b^2-4ac $$
es decir,
$$(2ar+b)^2 = b^2-4ac$$
por tanto
$$2ar+b = \pm \sqrt{b^2-4ac}$$
y por ende,
$$r = \frac{-b \pm \sqrt{b^2-4ac}}{2a}.$$\newpage\noindent
Nótese que $b^2-4ac$ es un número complejo, por lo que tiene dos raíces complejas $z$ y $w$, donde $w=-z$. Así que $\sqrt{b^2-4ac}$ representa a cualquiera, pero solo a una, de $z$ y $w$.

En resumen, las dos raíces (no necesariamente distintas) de la ecuación
$$ax^2+bx+c=0$$
vienen dadas por la fórmula
$$x=\frac{-b \pm \sqrt{b^2-4ac}}{2a}.$$

\section{El discriminante de la ecuación general de segundo grado}

Por la sección anterior, sabemos que las raíces de $ax^2+bx+c=0$ son
$$x_1=\frac{-b + \sqrt{b^2-4ac}}{2a} \qquad \text{ y } \qquad x_2=\frac{-b - \sqrt{b^2-4ac}}{2a}.$$
Por lo que el discriminante de dicha ecuación es
$$D=a^2(x_1-x_2)^2,$$
es decir:
$$D=b^2-4ac.$$

Observemos que si la ecuación de segundo grado $ax^2+bx+c=0$ es de coeficientes reales, entonces:
\begin{enumerate}[label=\roman*.]
    \item Tiene dos raíces reales si y solo si $D>0$.
    \item Tiene una raíz real doble si y solo si $D=0$.
    \item Tienen dos raíces imaginarias diferentes si y solo si $D<0$.
\end{enumerate}

\section{La ecuación de tercer grado} \label{sec:B4}

Consideremos la ecuación general de tercer grado
\begin{equation}
    Ax^3+Bx^2+Cx+D=0 \label{ecuaciongrado3}
\end{equation}
con $A$, $B$, $C$, $D \in \CC$ y $A \neq 0$ y con la incógnita $x$. Como $A \neq 0$, y puesto que la ecuación
$$\frac{1}{A} \left( Ax^3+Bx^2+Cx+D \right) =0$$
tiene las mismas raíces que la ecuación \eqref{ecuaciongrado3}, entonces no se pierde generalidad si en lugar de esta, escribimos
\begin{equation}
    x^3+bx^2+cx+d=0, \text{ con }  b,  c,  d \in \CC \label{ecuaciongrado3.1}
\end{equation}

Sustituyendo $\displaystyle x=y-\frac{b}{3}$ en \eqref{ecuaciongrado3.1}, se tiene que
$$\left( y-\frac{b}{3} \right)^3+b \left( y-\frac{b}{3} \right)^2+c \left( y-\frac{b}{3} \right) +d=0,$$
por tanto
$$y^3+\left( c-\frac{b^2}{3} \right) y + \left( d-\frac{bc}{3}+\frac{2b^3}{27} \right) =0.$$

\newpage

En consecuencia, resolver la ecuación \eqref{ecuaciongrado3.1} se reduce a resolver la ecuación
$$y^3+py+q=0$$
donde
$$p=c-\frac{b^2}{3} \qquad \text{ y } \qquad q=d-\frac{bc}{3}+\frac{2b^3}{27}.$$

Si $y_1$, $y_2$, $y_3$ son raíces de $y^3+py+q=0$, entonces:
$$x_1=y_1-\frac{b}{3}, \hspace{1cm} x_2=y_2-\frac{b}{3}, \hspace{1cm} x_3=y_3-\frac{b}{3}$$
son las raíces de \eqref{ecuaciongrado3.1}.

Resolveremos ahora la ecuación
\begin{equation}
    y^3+py+q=0. \label{ecuaciongrado3.2}
\end{equation}
Escribiendo $y=u+v$ en \eqref{ecuaciongrado3.2}, tenemos que
$$(u+v)^3+p(u+v)+q=0.$$
Desarrollando la expresión anterior, obtenemos
\begin{equation}
    u^3+v^3+q+(3uv+p)(u+v)=0. \label{ecuaciongrado3.3}
\end{equation}

Cualquiera que sea el valor numérico de la suma de $u$ y $v$, siempre podemos determinar a $u$ y $v$ imponiéndoles la condición adicional que su producto $uv$ sea un número prefijado. En efecto: supongamos que $u+v=r$ e impongamos la condición adicional $uv=s$, entonces $v=r-u$ y $uv=s$, por tanto $v=r-u$ y $u(r-u)=s$, por tanto $v=r-u$ y $u^2-r u+s=0$. En consecuencia $v=r-u$, y $u$ puede calcularse por la fórmula para resolver una ecuación de segundo grado. Esto demuestra lo que se afirmó.

Enseguida determinaremos a $u$ y $v$, sabiendo que la suma $u+v$ es raíz de la ecuación $y^3+p y+q=0$, e imponiendo la condición adicional
\begin{equation}
    uv=-\frac{p}{3}. \label{ecuaciongrado3.4}
\end{equation}
Sustituyendo \eqref{ecuaciongrado3.4} en \eqref{ecuaciongrado3.3}, tenemos que
\begin{equation}
    u^3+v^3+q=0. \label{ecuaciongrado3.5}
\end{equation}
De \eqref{ecuaciongrado3.4} y \eqref{ecuaciongrado3.5}, se sigue que
\begin{equation}
    u^3v^3=-\frac{p^3}{27} \label{ecuaciongrado3.6}
\end{equation}
y
\begin{equation}
    u^3+v^3=-q. \label{ecuaciongrado3.7}
\end{equation}

Puesto que
$$\left(z-u^3\right)\left(z-v^3\right)=z^2-\left(u^3+v^3\right)z+u^3v^3$$
entonces por \eqref{ecuaciongrado3.6} y \eqref{ecuaciongrado3.7}, $u^3$ y $v^3$ son las dos soluciones de la ecuación de segundo grado
\begin{equation}
    z^2+qz-\frac{p^3}{27}=0. \label{ecuaciongrado3.8}
\end{equation}

Por otro lado, las soluciones de la ecuación \eqref{ecuaciongrado3.8}, vienen dadas por
$$z_1=-\frac{q}{2}+\sqrt{\frac{q^2}{4}+\frac{p^3}{27}} \hspace{1cm} \text{y} \hspace{1cm} z_2=-\frac{q}{2}-\sqrt{\frac{q^2}{4}+\frac{p^3}{27}}.$$

\newpage

Por tanto, podemos escribir
\begin{equation}
    u^3=z_1 \hspace{1cm} \text{y} \hspace{1cm} v^3=z_2. \label{ecuaciongrado3.9}
\end{equation}

Nótese que las anteriores ecuaciones son del tipo $x^n=z$, por lo que se pueden resolver mediante el método expuesto en el \hyperref[FUNDAMENTAL]{Apéndice C}. Cada una de dichas ecuaciones tiene tres raíces, digamos $u_1$, $u_2$, $u_3$, para $u^3=z_1$; y $v_1$, $v_2$, $v_3$, para $v^3=z_2$.

Resolviendo la ecuación $x^3=1$, tenemos que sus raíces son:
$$w^0=1 \hspace{1cm} w=\frac{-1+\sqrt{3}i}{2} \hspace{1cm} w^2=\frac{-1-\sqrt{3}i}{2}$$

Es fácil comprobar que las raíces de
$$u^3=z_1$$
son:
$$u_1 \hspace{1cm} u_2=wu_1 \hspace{1cm} u_3=w^2u_1;$$
y las raíces de
$$v^3=z_2$$
son:
$$v_1 \hspace{1cm} v_2=wv_1 \hspace{1cm} v_3=w^2v_1.$$

No perdamos de vista que estamos determinando valores de $u$ y $v$, de modo que la suma $u+v$ sea raíz de $y^3+py+q=0$, y que además $u$ y $v$ satisfagan la condición adicional $\displaystyle uv=-\frac{p}{3}$. Hemos determinado tres posibles valores $u_1,  u_2$ y $u_3$ para $u$; y tres posibles valores $v_1,  v_2$ y $v_3$ para $v$.

Pero observemos que
$$\left(u_i v_j\right)^3=-\frac{p^3}{27}$$
no implica que
$$u_i v_j=-\frac{p}{3},$$
para cada $i$, $j = 1$, $2$, $3$. Si elegimos $u_1$ y $v_1$ de modo que $\displaystyle u_1v_1=-\frac{p}{3}$, y a los que denotaremos por
\begin{equation}
    u_1=\sqrt[3]{\frac{-q}{2}+\sqrt{\frac{q^2}{4}+\frac{p^3}{27}}} \label{APPCOSIPA1}
\end{equation}
y
\begin{equation}
    v_1=\sqrt[3]{\frac{-q}{2}-\sqrt{\frac{q^2}{4}+\frac{p^3}{27}}}. \label{APPCOSIPA2}
\end{equation}

Entonces las raíces de
$$y^3+py+q=0$$
son
\begin{equation}
    y_1=u_1+v_1 \hspace{1cm} y_2=wu_1+w^2v_1 \hspace{1cm} y_3=w^2u_1+wv_1. \label{APPCOSIPA3}
\end{equation}
Las expresiones anteriores son conocidas como \emph{fórmulas de Cardano}, para calcular las raíces de la ecuación
$$y^3+py+q=0.$$

\newpage

\section{El discriminante de la ecuación general de tercer grado}\label{disc_tercergrado}

De acuerdo a las fórmulas de Cardano, y a la definición \ref{definicion:B.1.1}, el discriminante de la ecuación $y^3+py+q=0$ es
$$D=(y_1-y_2)^2(y_1-y_3)^2(y_2-y_3)^2.$$

Recordemos que
$$w=-\frac{1}{2}+\frac{1}{2}\sqrt{3}i$$
es raíz de la ecuación $x^3-1=0$, por tanto
$$w^3=1 \hspace{0.5cm} \text{y} \hspace{0.5cm} w^2+w+1=0$$
entonces
\begin{align*}
    y_1-y_2 &=(u_1+v_1)-\left(wu_1+w^2v_1\right) \\
    &=(1-w)u_1-w^2v_1+w^3v_1 \\
    &=(1-w)u_1+(1-w)\left(-w^2v_1\right) \\
    &=(1-w)\left(u_1-w^2v_1\right), \\
    & \\
    y_1-y_3 &=(u_1+v_1)-\left( w^2u_1+wv_1 \right) \\
    &=\left( 1-w^2 \right) u_1-wv_1+w^3v_1 \\
    &=\left( 1-w^2 \right) u_1+\left( 1-w^2 \right) (-wv_1) \\
    &=\left( 1-w^2 \right) (u_1-wv_1), \\
    & \\
    y_2-y_3 &=\left( wu_1+w^2v_1 \right) - \left( w^2u_1+wv_1 \right) \\
    &=\left( w-w^2 \right) u_1-wv_1+w^2v_1 \\
    &=\left( w-w^2 \right) u_1+\left( w-w^2 \right) (-v_1) \\
    &=\left( w-w^2 \right) (u_1-v_1).
\end{align*}

Además
\begin{align*}
    (1-w)\left( 1-w^2 \right) &= \left( \frac{3}{2}-\frac{\sqrt{3}}{2} i \right) \left( \frac{3}{2}+\frac{\sqrt{3}}{2} i \right) \\
    &=3
\end{align*}
y
$$w-w^2=\sqrt{3}i.$$
Puesto que
$$(x-1)(x-w)\left( x-w^2 \right) =x^3-1,$$
entonces
$$\left( \frac{u_1}{v_1} -1 \right) \left( \frac{u_1}{v_1} -w \right) \left( \frac{u_1}{v_1}-w^2 \right) = \left( \frac{u_1}{v_1} \right)^3 -1$$
por tanto
\begin{equation}
    (u_1-v_1)(u_1-wv_1)\left( u_1-w^2v_1 \right) = u^3_1-v^3_1 \label{onono}
\end{equation}
Por \eqref{ecuaciongrado3.9} de la sección \ref{sec:B4}, de \eqref{onono} se sigue que
$$(u_1-v_1)(u_1-wv_1)\left( u_1-w^2v_1 \right) = 2 \sqrt{\frac{q^2}{4}+\frac{p^3}{27}}.$$

\newpage\noindent
En consecuencia,
\begin{align*}
    D &=\left[ (y_1-y_2)(y_1-y_3)(y_2-y_3) \right]^2 \\
    &=\left[ (1-w)\left( u_1-w^2v_1 \right) \left( 1-w^2 \right) (u_1-wv_1) \left( w-w^2 \right) (u_1-v_1) \right]^2 \\
    &=\left[ (1-w) \left( 1-w^2 \right) \left( w-w^2 \right) (u_1-v_1) (u_1-wv_1) \left( u_1-w^2v_1 \right) \right]^2 \\
    &=\left[ 3 \left( \sqrt{3}i \right) \left( 2 \sqrt{\frac{q^2}{4}+\frac{p^3}{27}} \right) \right]^2 \\
    &= -108 \left( \frac{q^2}{4} + \frac{p^3}{27} \right) \\
    &=-27q^2-4p^3
\end{align*}

En resumen, el discriminante de la ecuación
\begin{equation}
    y^3+py+q=0 \label{TOCOF}
\end{equation}
es
$$D=-4p^3-27q^2.$$
Si $y_1$, $y_2$, $y_3$ son las raíces de \eqref{TOCOF}, donde
$$p=c-\frac{b^2}{3}$$
y
$$q=d-\frac{bc}{3}+\frac{2b^3}{27}$$
se sabe que
$$x_1 = y_1-\frac{b}{3} \hspace{1cm} x_2 = y_2-\frac{b}{3} \hspace{1cm} x_3 = y_3-\frac{b}{3}$$
son las raíces de
$$x^3+bx^2+cx+d=0.$$
Por tanto
$$x_1-x_2=y_1-y_2 \hspace{1cm} x_1-x_3=y_1-y_3 \hspace{1cm} x_2-x_3=y_2-y_3.$$

En consecuencia, el discriminante de
$$x^3+bx^2+cx+d=0$$
es
$$\Delta =-4p^3-27q^2,$$
es decir,
\begin{align*}
    \Delta & = -4 \left( c - \frac{b^2}{3} \right)^3 - 27 \left( d - \frac{bc}{3} + \frac{2b^3}{27} \right)^2 \\
    & =18bcd-4b^3d+b^2c^2-4c^3-27d^2.
\end{align*}

Consideremos el escenario en el cual la ecuación cúbica
$$x^3+bx^2+cx+d=0$$
posee \emph{coeficientes reales} (y por tanto, también su ecuación asociada $y^3 + py + q=0$ tiene coeficientes reales). Dado que esta ecuación es de grado impar, garantiza la existencia de al menos una raíz real. Por lo tanto, se presentan las siguientes tres posibilidades en relación con sus raíces:\newpage
\begin{enumerate}[label=\roman*.]
    \item Si $\Delta >0$, entonces tiene una raíz real y dos raíces complejas conjugadas.
    \item Si $\Delta =0$, entonces tiene tres raíces reales y al menos dos de ellas son iguales.
    \item Si $\Delta <0$, entonces tiene tres raíces reales diferentes.
\end{enumerate}

\begin{examplebox}{}{ejercicioimportantee}
    Resuelva la ecuación
    $$y^3+3y^2-3y-14=0.$$

    \tcblower
    \solucion Usando la sustitución $y=x-1$ reducimos esta ecuación a la forma
    $$x^3-6x-9=0.$$
    Notemos que $p=-6$ y $q=-9$, por lo cual
    $$\frac{q^2}{4}+\frac{p^3}{27}=\frac{49}{4}>0.$$
    Entonces, la ecuación
    $$x^3-6x-9=0$$
    tiene una raíz real y dos raíces complejas conjugadas. Según las expresiones \eqref{APPCOSIPA1} y \eqref{APPCOSIPA2},
    $$u_1=\sqrt[3]{\frac{9}{2}+\frac{7}{2}}=\sqrt[3]{8} \quad \text{ y } \quad v_1=\sqrt[3]{\frac{9}{2}-\frac{7}{2}}=\sqrt[3]{1}.$$
    Por consiguiente, $u_1=2$ y $v_1=1$, es decir, $x_1=3$. Las otras dos raíces se hallan por las fórmulas \eqref{APPCOSIPA3}:
    $$x_2 = - \frac{3}{2} + \frac{\sqrt{3}}{2}i \quad \text{ y } \quad x_2 = - \frac{3}{2} - \frac{\sqrt{3}}{2}.$$
    De aquí se deduce que las raíces de la ecuación dada son:
    $$y_1 = 2, \quad y_2 = - \frac{5}{2} + \frac{\sqrt{3}}{2}i, \quad y_3 = - \frac{5}{2} - \frac{\sqrt{3}}{2}i.$$
\end{examplebox}

\begin{examplebox}{}{}
    Resuelva la ecuación
    $$x^3 - 9x^2 + 36x - 80 = 0.$$

    \tcblower
    \solucion Usando la sustitución $x = y + 3$ reducimos esta ecuación a la forma
    $$y^3 + 9y - 26 = 0.$$
    Notemos que $p=9$ y $q=-26$, por lo cual
    $$\frac{q^2}{4}+\frac{p^3}{27}=196>0.$$
    Entonces, la ecuación
    $$y^3 + 9y - 26 = 0$$
    tiene una raíz real y dos raíces complejas conjugadas. Según las expresiones \eqref{APPCOSIPA1} y \eqref{APPCOSIPA2},
    $$u_1=\sqrt[3]{13 + \sqrt{196}}=\sqrt[3]{27} \quad \text{ y } \quad v_1=\sqrt[3]{13 - \sqrt{196}}=\sqrt[3]{-1}.$$
    Por consiguiente, $u_1=3$ y $v_1=-1$, es decir, $y_1=2$. Las otras dos raíces se hallan por las fórmulas \eqref{APPCOSIPA3}:
    $$y_2=-1+2\sqrt{3}i, \quad y_3=-1-2\sqrt{3}i.$$
    De aquí se deduce que las raíces de la ecuación dada son:
    $$x_1=5, \quad x_2=2+2\sqrt{3}i, \quad x_3=2-2\sqrt{3}i.$$
\end{examplebox}

\newpage

\begin{examplebox}{}{}
    Resuelva la ecuación
    $$x^3-12x+16=0.$$

    \tcblower
    \solucion Notemos que $p=-12$ y $q=16$, por lo tanto
    $$\frac{q^2}{4}+\frac{p^3}{27}=0.$$
    De esta ecuación, obtenemos $u_1=\sqrt[3]{-8}$, lo que implica que $u_1=-2$. En consecuencia, las raíces son $x_1=-4$ y $x_2=2=x_3$.
\end{examplebox}

\section{La ecuación de cuarto grado}

Consideremos la ecuación general de cuarto grado,
\begin{equation}
    Ax^4+Bx^3+Cx^2+Dx+E=0, \label{ecuacioncuarto7.1}
\end{equation}
donde $A,  B,  C,  D,  E \in \CC$ y $A \neq 0$, con la incógnita $x$.

Como $A \neq 0$, y puesto que la ecuación
$$\frac{1}{A} \left( Ax^4+Bx^3+Cx^2+Dx+E \right) =0$$
tiene las mismas raíces que la ecuación \eqref{ecuacioncuarto7.1}, entonces no se pierde generalidad si en lugar de esta, escribimos
\begin{equation}
    x^4+ax^3+bx^2+cx+d=0 \label{ecuacioncuarto7.2}
\end{equation}

Si la ecuación \eqref{ecuacioncuarto7.2} la escribimos en la forma
$$x^4+ax^3=-bx^2-cx-d$$
y sumamos $\displaystyle \frac{a^2}{4}x^2$ a ambos miembros de la ecuación, entonces
$$x^4+ax^3+\frac{a^2}{4}x^2=\frac{a^2}{4}x^2-bx^2-cx-d,$$
donde se sigue que
\begin{equation}
    \left( x^2+\frac{a}{2}x \right)^2 = \left( \frac{a^2}{4}-b \right) x^2-cx-d. \label{ecuacioncuarto7.3}
\end{equation}

Si el miembro derecho de la ecuación \eqref{ecuacioncuarto7.3} fuera un cuadrado perfecto, es decir, si fuera de la forma $(ex+f)^2$, entonces resolver dicha expresión y resolver \eqref{ecuacioncuarto7.2} sería inmediato. Pero en general dicho miembro derecho no es un cuadrado perfecto.

Sumando
$$\left( x^2+\frac{a}{2}x \right) y + \frac{y^2}{4}$$
a ambos miembros de \eqref{ecuacioncuarto7.3}, tenemos que
$$\left( x^2+\frac{a}{2}x \right)^2 + \left( x^2+\frac{a}{2}x \right) y + \frac{y^2}{4} = \left( \frac{a^2}{4}-b \right) x^2-cx-d + \left( x^2+\frac{a}{2}x \right) y + \frac{y^2}{4},$$
por tanto
\begin{equation}
    \left( x^2+\frac{a}{2}x+\frac{y}{2} \right)^2 = \left( \frac{a^2}{4}-b+y \right) x^2 + \left( -c+\frac{1}{2} ay \right) x+ \left( -d + \frac{1}{4} y^2 \right). \label{ecuacioncuarto7.4}
\end{equation}

\newpage

Vamos a determinar $y$ de modo que el miembro derecho de la expresión \eqref{ecuacioncuarto7.4} sea un cuadrado perfecto. Para esto observemos que
$$\alpha x^2+\beta x+\gamma =(ex+f)^2,$$
si y solo si
$$\beta ^2-4\alpha \gamma =0.$$
En efecto
\begin{align*}
    \alpha x^2+\beta x+\gamma =(ex+f)^2 & \Longrightarrow \alpha x^2+\beta x+\gamma =e^2x^2+2efx+f^2 \\ 
    & \Longrightarrow \alpha =e^2, \; \beta =2ef, \; \gamma =f^2 \\ 
    & \Longrightarrow \beta ^2-4\alpha \gamma =0.
\end{align*}
Recíprocamente, si $\beta ^2-4\alpha \gamma =0$, entonces
\begin{align*}
    \alpha x^2+\beta x+\gamma  &=\alpha  \left( x^2+\frac{\beta }{\alpha }x+\frac{\gamma }{\alpha } \right) \\
    &=\alpha  \left( x+\frac{\beta }{2\alpha }-\frac{\sqrt{\beta ^2-4\alpha \gamma }}{2\alpha } \right) \left( x+\frac{\beta }{2\alpha }+\frac{\sqrt{\beta ^2-4\alpha \gamma }}{2\alpha } \right) \\
    &=\alpha  \left( x+\frac{\beta }{2\alpha } \right)^2 \\
    &=\left( \sqrt{\alpha }x+\frac{\beta }{2\sqrt{\alpha }} \right)^2 \\
    &=(ex+f)^2
\end{align*}
con $e=\sqrt{\alpha }$ y $\displaystyle f=\frac{\beta }{2\sqrt{\alpha }}$. En consecuencia, el miembro derecho de \eqref{ecuacioncuarto7.4} será un cuadrado perfecto, es decir,
$$\left( \frac{a^2}{4}-b+y \right) x^2+\left( -c+\frac{1}{2}ay \right) x + \left( -d+\frac{1}{4}y^2 \right) = (ex+f)^2,$$
si y solo si
$$\left( -c+\frac{1}{2}ay \right)^2 -4 \left( \frac{a^2}{4}-b+y \right) \left( -d+\frac{1}{4}y^2 \right) =0,$$
si y solo si
$$c^2-acy+\frac{1}{4}a^2y^2-4 \left( -\frac{a^2d}{4}+\frac{a^2y^2}{16}+bd-\frac{1}{4}by^2-dy+\frac{1}{4} y^3 \right) =0,$$
si y solo si
$$-y^3+by^2+(4d-ac)y+a^2d-4bd+c^2=0,$$
si y solo si
\begin{equation}
    y^3-by^2+(ac-4d)y+\left(4bd-a^2d-c^2\right)=0. \label{ecuacioncuarto7.5}
\end{equation}

Si $y_0$ es una solución cualquiera de la ecuación cúbica \eqref{ecuacioncuarto7.5}, la cual es llamada la \textit{resolvente} de la ecuación \eqref{ecuacioncuarto7.2} de grado cuatro, entonces por \eqref{ecuacioncuarto7.4} tenemos que:
$$\left( x^2+\frac{a}{2}x+\frac{y_0}{2} \right)^2 = (ex+f)^2$$
por tanto
\begin{equation}
    x^2+\frac{a}{2}x+\frac{y_0}{2}=ex+f \label{ecuacioncuarto7.6}
\end{equation}
y
\begin{equation}
    x^2+\frac{a}{2}x+\frac{y_0}{2}=-ex-f. \label{ecuacioncuarto7.7}
\end{equation}

Las cuatro soluciones de \eqref{ecuacioncuarto7.6} y \eqref{ecuacioncuarto7.7} son las raíces de \eqref{ecuacioncuarto7.2}.

\newpage\noindent

\begin{adjustwidth}{-7.6cm}{-2cm}
    \begin{tcolorbox}[
        theorem style=change break,
        enhanced,
        breakable,
        boxrule=0pt,
        frame hidden,
        left = 1.8cm,
        right = 1.8cm,
        top=4mm,
        bottom=6mm,
        colback=black!7!white,
        coltitle=black,
        attach title to upper={\ },
        sharp corners,
        borderline north={1.5pt}{0pt}{black},
        title = {Algoritmo para resolver una ecuación de cuarto grado:},
        fonttitle=\selectfont\Lato\bfseries\LARGE,
        fontupper=\normalsize
    ]
        \begin{multicols}{2}
            \begin{itemize}
                \item \textbf{Paso 1:} Divida toda la ecuación entre $A$. Esto transforma la ecuación en:
                $$x^4 + ax^3 + bx^2 + cx + d = 0.$$
                \item \textbf{Paso 2:} Sustituya y obtenga la expresión:
                $$\left( x^2 + \frac{a}{2}x \right)^2 = \left( \frac{a^2}{4} - b \right) x^2 - cx - d.$$
                \item \textbf{Paso 3:} Verifique si el lado derecho de la ecuación es un cuadrado perfecto de la forma $(ex + f)^2$. Si es así, resolver la ecuación principal es inmediato. En caso contrario, forme la ecuación:
                \begin{align*}
                    \left( x^2 + \frac{a}{2}x + \frac{y}{2} \right)^2 & = \left( \frac{a^2}{4} - b + y \right) x^2 \\
                    & + \left(- c + \frac{1}{2} ay \right) x + \left(- d + \frac{1}{4} y^2 \right).
                \end{align*}
                %\columnbreak
                \item \textbf{Paso 4:} Resuelva la ecuación cúbica auxiliar:
                $$y^3 - by^2 + (ac - 4d)y + \left(4bd - a^2d - c^2\right) = 0.$$
                \item \textbf{Paso 5:} Si $y_0$ es una solución cualquiera de la ecuación cúbica auxiliar, basta encontrar las raíces de
                $$x^2 + \frac{a}{2}x + \frac{y_0}{2} = ex + f \quad \text{y} \quad x^2 + \frac{a}{2}x + \frac{y_0}{2} = -ex - f$$
                que son las soluciones de la ecuación principal, recordando que $e = \sqrt{\alpha}$ y $f = \dfrac{\beta}{2\sqrt{\alpha}}$ con
                $$\alpha = \frac{a^2}{4} - b + y_0 \quad \text{y} \quad \beta = - c + \frac{1}{2} ay_0.$$
            \end{itemize}
        \end{multicols}
    \end{tcolorbox}
\end{adjustwidth}

\begin{examplebox}{}{}
    Resuelva la ecuación
    $$4x^4 + 4x^3 - 12x^2 + 4x + 4 = 0.$$

    \tcblower
    \solucion Dividiendo entre 4, la ecuación toma la forma
    $$x^4 + x^3 - 3x^2 + x + 1 = 0.$$
    Entonces, podemos reescribir la ecuación en la forma \eqref{ecuacioncuarto7.3}. En nuestro caso,
    $$\left( x^2 + \frac{1}{2}x \right)^2 = 4x^2 - x - 1.$$
    Lastimosamente, el lado derecho de la ecuación anterior no es un cuadrado perfecto. Entonces debemos reescribir nuevamente la ecuación a la forma \eqref{ecuacioncuarto7.4}. Así,
    $$\left( x^2 + \frac{1}{2}x + \frac{y}{2} \right)^2 = \left(\frac{13}{4} + y\right)x^2 + \left( -1 + \frac{1}{2}y \right)x + \left( - 1 + \frac{1}{4} y^2 \right).$$
    Ahora debemos resolver la ecuación \eqref{ecuacioncuarto7.5}, en nuestro caso, debemos resolver
    $$y^3 + 3y^2 - 3y - 14 = 0.$$
    Del ejemplo \ref{examplebox:ejercicioimportantee}, sabemos que una raíz es $y = 2$. Ahora, debemos resolver las ecuaciones \eqref{ecuacioncuarto7.6} y \eqref{ecuacioncuarto7.7}, es decir,
    $$x^2 + \frac{1}{2}x + 1 = \sqrt{\frac{21}{4}}x$$
    y
    $$x^2 + \frac{1}{2}x + 1 = - \sqrt{\frac{21}{4}}x.$$
    Usando la fórmula general, obtenemos las soluciones de las anteriores ecuaciones respectivamente,
    $$x_1 = \frac{-1 + \sqrt{21}}{4} + \frac{\sqrt{2\sqrt{21} - 6}}{4}i, \quad x_2 = \frac{-1 + \sqrt{21}}{4} - \frac{\sqrt{2\sqrt{21} - 6}}{4}i$$
    y
    $$x_3 = \frac{-1 + \sqrt{21} - \sqrt{2\sqrt{21} + 6}}{4}, \quad x_4 = \frac{-1 + \sqrt{21} + \sqrt{2\sqrt{21} + 6}}{4}$$
    que son justo, las raíces de la ecuación principal dada.
\end{examplebox}

\newpage

\begin{examplebox}{}{}
    Resuelva la ecuación
    $$4x^4 + 12x^2 + 4i\sqrt{10}x - 3 = 0.$$

    \tcblower
    \solucion Dividiendo entre 4, la ecuación toma la forma
    $$x^4 + 3x^2 + i\sqrt{10}x - \frac{3}{4} = 0.$$
    Entonces, podemos reescribir la ecuación en la forma \eqref{ecuacioncuarto7.3}. En nuestro caso,
    $$\left( x^2 \right)^2 = - 3x^2 - i\sqrt{10}x + \frac{3}{4}.$$
    Lastimosamente, el lado derecho de la ecuación anterior no es un cuadrado perfecto. Entonces debemos reescribir nuevamente la ecuación a la forma \eqref{ecuacioncuarto7.4}. Así,
    $$\left( x^2 + \frac{y}{2} \right)^2 = \left(-3 + y\right)x^2 + -i\sqrt{10}x + \left( \frac{3}{4} + \frac{1}{4} y^2 \right).$$
    Ahora debemos resolver la ecuación \eqref{ecuacioncuarto7.5}, en nuestro caso, debemos resolver
    $$y^3 - 3y^2 + 3y + 1 = 0.$$
    Usando la sustitución $y = z + 1$ reducimos esta ecuación a la forma
    $$z^3 + 2 = 0.$$
    Notemos que $p = 0$ y $q = 2$, por lo cual
    $$\frac{q^2}{4} + \frac{p^3}{27} = 1 > 0.$$
    Entonces, la ecuación
    $$z^3 + 2 = 0$$
    tiene una raíz real y dos raíces complejas conjugadas. Según las expresiones \eqref{APPCOSIPA1} y \eqref{APPCOSIPA2},
    $$u_1 = \sqrt[3]{\frac{-2}{2} + \sqrt{1}} \quad \text{ y } \quad v_1 = \sqrt[3]{\frac{-2}{2} - \sqrt{1}}.$$
    Por consiguiente, $u_1 = 0$ y $v_1 = -\sqrt[3]{2}$, es decir, $y = 1 -\sqrt[3]{2}$. Ahora, debemos resolver las ecuaciones \eqref{ecuacioncuarto7.6} y \eqref{ecuacioncuarto7.7}, es decir,
    $$x^2 + \frac{1 - \sqrt[3]{2}}{2} = \sqrt{-2 + \sqrt[3]{2}}x - \frac{i\sqrt{10}}{2\sqrt{-2 + \sqrt[3]{2}}}$$
    y
    $$x^2 + \frac{1 - \sqrt[3]{2}}{2} = - \sqrt{-2 + \sqrt[3]{2}}x + \frac{i\sqrt{10}}{2\sqrt{-2 + \sqrt[3]{2}}}.$$
    Usando la fórmula general, obtenemos las soluciones de las anteriores ecuaciones respectivamente,
    $$x_{1, 2} = \frac{1}{2}i \sqrt{2 + \sqrt[3]{2}} \pm \frac{1}{2}i \sqrt{4 - \sqrt[3]{2} + 2\sqrt{\frac{10}{2 + \sqrt[3]{2}}}}$$
    y
    $$x_{3, 4} = \pm \frac{1}{2} \sqrt{- 4 + \sqrt[3]{2} + 2\sqrt{\frac{10}{2 + \sqrt[3]{2}}}} - \frac{1}{2}i\sqrt{2 + \sqrt[3]{2}}$$
    que son justo, las raíces de la ecuación principal dada.
\end{examplebox}

\newpage

\section{La ecuación de quinto grado}\label{ecuacion_quinto_grado}

Las fórmulas para la resolución de las ecuaciones de tercero y cuarto grado fueron descubiertas ya en el siglo XVI. Al mismo tiempo comenzaron las búsquedas de fórmulas para la resolución de las ecuaciones de quinto grado y de grados superiores. Señalemos que la forma general de una ecuación de $n$-ésimo grado es:
$$a_nx^n+a_{n-1}x^{n-1}+\cdots +a_1 x+a_0 = 0$$
Estas búsquedas continuaron sin éxito hasta comienzos del siglo XIX, cuando por fin fue demostrado el siguiente resultado extraordinario:
\begin{tcolorbox}[
    % Estilo del teorema
    theorem style=change apart,
    enhanced,
    lower separated=false,
    breakable,
    % Bordes
    boxrule=0pt,
    frame hidden,
    % Colores
    colback=black!7!white,
    coltitle=black,
    % Título
    boxed title style={colframe=white, colback=white, boxrule=0pt},
    % Texto
    fontupper=\normalsize,
    before upper={\abovedisplayskip=8pt\belowdisplayskip=8pt},
    % Márgenes
    left=1mm,
    right=1mm,
    top=1mm,
    bottom=1mm,
    % Esquinas
    sharp corners,
]
    Para ningún $n$, mayor o igual a cinco puede hallarse formula que exprese las raíces de cualquiera ecuación de $n$-ésimo grado mediante sus coeficientes por radicales.
\end{tcolorbox}

Más aún, para cualquier $n$ mayor o igual a cinco se puede indicar una ecuación de $n$-ésimo grado con coeficientes enteros, cuyas raíces no pueden expresarse mediante radicales. Tal es, por ejemplo, la ecuación
$$x^5 - 4x - 2 = 0.$$
Puede demostrarse que esta ecuación tiene cinco raíces, tres reales y dos complejas, pero ninguna de ellas puede expresarse mediante radicales, es decir, esta ecuación es “irresoluble por radicales”. De este modo la reserva de números, reales o complejos que son raíces de las ecuaciones con coeficientes enteros (estos números se denominan algebraicos en contraposición a los números trascendentes que no son raíces de ninguna ecuación con coeficientes enteros), es mucho más amplia que la reserva de números que se expresan por radicales.

La inexistencia de fórmulas generales para la resolución por radicales de las ecuaciones de $n$-ésimo grado cuando $n \geq 5$ fue demostrada por Abel (1802-1829). La existencia de ecuaciones con coeficientes enteros irresolubles por radicales fue establecida por Galois (1811-1832), quien también halló las condiciones en las cuales la ecuación puede resolverse por radicales. Todos estos resultados exigieron la creación de una nueva y profunda teoría, la teoría de grupos. El concepto de grupo permitió agotar la cuestión referente a la resolución de ecuaciones por radicales, habiendo hallado más tarde numerosas aplicaciones en diferentes ramas de la matemática y fuera de sus límites, convirtiéndose en uno de los objetos más importantes de estudio en el álgebra.

El hecho de que no existen fórmulas para resolver las ecuaciones de $n$-ésimo grado cuando $n \geq 5$ no provoca dificultades serias en lo que respecta a la búsqueda práctica de las raíces de las ecuaciones. Esto se compensa totalmente por los numerosos métodos de resolución aproximada de las ecuaciones, que incluso en el caso de las ecuaciones cúbicas conducen al objetivo con mayor rapidez que utilizando la fórmula y extrayendo, a continuación, en forma aproximada los radicales reales. No obstante, la existencia de fórmulas para las ecuaciones de segundo, tercero y cuarto grados permitió demostrar que estas ecuaciones poseen respectivamente dos, tres o cuatro raíces.

En este apéndice, nos hemos referido aquí solamente a las ecuaciones de un cierto grado con una incógnita. El origen de esta teoría se remonta al álgebra elemental, en la que luego del estudio de ecuaciones de primer grado, se pasan al estudio de ecuaciones cuadráticas. Pero en el álgebra elemental también se dio un paso en otra dirección: luego de estudiar una ecuación de primer grado con una incógnita se pasó a considerar el sistema de dos ecuaciones de primer grado con dos incógnitas y el sistema de tres ecuaciones con tres incógnitas. El desarrollo de esta teoría se da en el curso de Álgebra II dado en la ESFM. En el mismo se estudian los métodos de resolución de cualesquiera sistemas de $n$ ecuaciones de primer grado con $n$ incógnitas, así como también los métodos para hallar la solución de aquellos sistemas de ecuaciones de primer grado en los cuales el número de ecuaciones no es igual al número de incógnitas. La teoría de los sistemas de ecuaciones de primer grado, así como otras, que le son afines, en particular, la teoría de matrices, forman una rama especial del álgebra, el álgebra lineal; de entre todas las partes del álgebra, ésta es la principal debido a sus aplicaciones en geometría y otras ramas de las matemáticas, así como también en física y mecánica teórica.

Por otra parte, tanto la teoría de las ecuaciones algebraicas como el álgebra lineal, en gran medida, pueden ser considerados actualmente como partes acabadas de la ciencia. Las necesidades de ramas contiguas de la matemática y la física condujeron a que en el álgebra pasó a primer plano el estudio de los conjuntos en los cuales están dadas las operaciones algebraicas. Además de la teoría de los campos, dentro de la cual entran la teoría de los números algebraicos y la teoría de las funciones algebraicas, ahora también se desarrolla la teoría de los anillos.